    Dans les améliorations technologique à apporter à la direction matériel, il nous faut, non seulement un moyen de gérer le matériel, mais aussi un moyen de communiquer et d'accéder à cette gestion sur les chantiers.

    Comme les différents sites ne seront pas proches géographiquement, il faudra mettre en place un réseau privé virtuel entre les différents sites, et avec les clients mobiles (smartphone et site chantier).

    Les différents sites et le sièges profiteront d'un accès câblé à l'internet, il n'y auras donc aucune difficulté à mettre en place un réseau privé virtuel.
    En revanche pour les sites de chantiers, l'idéal serait de profiter d'un accès câblé à l'internet, mais ce ne sera pas toujours possible.
    On peut penser alors à un accès par réseaux 3G/EDGE.
    Pour ces dernier cas, il serait préférable de mettre en place, comme pour les sites, une base de donnée locale afin d'apporter les modifications et de les porter ensuite vers le serveur globales, minimisant ainsi la quantité de données à envoyer.

    Le siège accueillera donc le serveur de données principale, ainsi qu'un serveur de sauvegarde pour éviter toute perte, et pouvoir redémarrer rapidement la production en cas de panne.
    Les différents sites fixes accueillerons également un base de données secondaire, plus petite, permettant uniquement d'enregistrer des modifications en cas d'inaccessibilité au serveur central.

    Pour rester opérationnel sur les chantier non couverts par le réseaux 3G/EDGE, il serait envisageable de mettre en place des relais Wifi faisant la connexion entre le réseaux interne du site de chantier, et les appareils mobiles sur le terrains.

    Pour permettre aux utilisateurs de manipuler cette base de donnée, il faudra leur fournir chacun un poste client, ainsi qu'un réseaux local de connexion, et quelques services d'impressions.

\section{Mise en place de l'infrastructure de communication}

    Notre solution de communication se basera principalement sur l'utilisation d'un VPN.

    Chaque site accueil déjà un serveur d'application, ce serveur auras à sa charge un serveur VPN fournissant à tout les clients un accès de type intranet identique sur tout les site.
    Les terminaux mobiles auront également accés à ce VPN par le biais de connection mobile, ou par wifi.

    Tous les clients se connecterons donc à ces serveurs à l'aide de clients VPN et profiterons ainsi d'un acces sécurisé à toutes les données necessaires.

    Pour ce qui est du reseaux, l'installation sera principalement impacté par la disposition des locaux.
    Dans tout les cas, on ne pourra se passer, par site, d'un firewall et d'un routeur, suivi d'un ou plusieurs switch permettant de distribuer le reseaux dans chacune des infrastructures.

    Etant donné que la pluspart des employé auront une grande mobilité, il peut être judicieux de les équiper uniquement de postes clients portables.


    \subsection{Siége}

        \subsubsection{Serveur d'application / de données}
            Le siège comporteras la base de données principale. Elle se doit d'être donc la plus performante possible, et résister à la charge.

            Le besoin en stoquage et en réactivité ne justifie pas l'utilisation de cluster, ni de base de données disjointe.
            Le choix le plus judicieux est donc un serveur unique performant et disposant d'une grande quantité de stoquage.
            Pour une estimation du prix en vue d'un choix précis, on peut le borner très généreusement à 10 000€.

            Le serveur sera de type rack et sera installé dans une salle dédiée réfrigérée avec accès réglementé.
            Cette salle comportera également, le firewall, le routeur et le switch.

            Ce serveur sera le coeur de toutes l'installation, autrement dit il sera le plus critique, il est important qu'il soit de la meilleur qualité possible.
            Il permettras l'execution du serveur VPN ainsi que de l'application.

        \subsubsection{Réseau : Firewall, Routeur et switch}
            Afin de connecter les postes clients aux serveur, l'installation d'un réseau local est indispensable.
            Cette connection se fera par l'intermediaire d'un switch sur lequel le serveur et tous les postes clients se connecterons.
            Un routeur en amont permettras notamment la distribution d'adresse IP.

            Une estimation du prix du routeur et du switch pourras être chacun de 1000€.
            
            Afin de connecter ce réseau local à l'exterieur, l'accès internet sera fourni par l'intermédiaire d'un firewall.
            Ce firewall sera de qualité professionnel, et sera aussi important que le serveur lui-même.
    
            Son prix pourras être estimé à 1000€.

        \subsubsection{Postes clients}
            Cette architecture permettra de connecter autant de postes clients que necessaires.
            Le nombre exacte, et le materiel des postes clients sera détaillé plus loin.

    \subsection{Sites secondaires}

        \subsubsection{Serveur d'application / de données}
            Tout comme le siège chacun des sites secondaire comportera un serveur hebergeant l'application, et un clone de la base de données du siège.

            Ces sites étant moins important en effectif, la charge sur le serveur sera donc moindre.
            Cependant, on peut s'attendre à une expansion de l'entreprise, il serait donc judicieux d'investir dans du materiel performant, sans être inutilement couteux.
            
            Le choix le plus judicieux sera probablement une tour simple dans une pièce unique réfrigéré avec accès reglementé.
            Comme sur le site principal, cette pièce comporteras le firewall, le routeur et le switch.

        \subsubsection{Réseau : Firewall, Routeur et switch}
            L'infrastucture sera sensiblement la même que sur le site principale, je ne la détaillerais donc pas ici.

    \subsection{Chantier}
                    
            Le manque d'information concernant la connectivité, la couverture réseaux, et le personnel sur le chantier, entrave cette étude.

            Les chantiers peuvent être considéré comme des sites secondaires par rapport à la relation avec le server.
            Ils seront donc équipés du même type de serveur que les sites secondaires, et du même type d'architecture réseaux.

            En revanche, on y ajoutera un relais wifi couvrant la zone du chantier permettant une connexion avec les Clients mobiles.
   
        \subsubsection{Clients mobiles}
            Les employés amenés à se déplacer réguliérement sur le chantier, pourront bénéficier à tout moment d'un accès à l'application de gestion par le biais d'un smartphone connecté au vpn de l'entreprise.
            Ils pourront ainsi, à l'aide de l'interface web, continuer à travailler pendant leurs déplacement, ou sur le chantier.
            Le nombre de smartphone influe peu sur l'infrastructure global, il pourras être ajusté aux besoins de l'entreprise.

\section{Liste du matériel}

    \subsection{Serveur et réseaux}
        \subsubsection{Siège}
            \begin{itemize}
	            \item 1 serveur d'application / de données important
                \item 1 firewall
	            \item 1 routeur 
                \item 1 switch
                \item suffisament de câble
            \end{itemize}

        \subsubsection{Sites distants}
            \begin{itemize}
	            \item 4 serveurs d'application / de données
                \item 4 firewall par site
	            \item 4 routeurs par site
                \item 4 switch
                \item suffisament de câble
            \end{itemize}

            \begin{itemize}
	            \item 40 serveur d'application / données mobile
                \item 40 firewall
	            \item 40 routeur
	            \item 40 switch
	            \item 40 relais wifi
                \item suffisament de câble
            \end{itemize}

   \subsection{Postes clients}

        TODO : demander au responsable métier le nombre de postes à fournir.

        \subsubsection{Direction Materiel}
            60 postes existants ??? 60 personnes ???

        \subsubsection{Département Matériel}
            5 personnes.

            Existant :
                \begin{itemize}
                    \item 3 postes
                    \item 2 imprimantes
                \end{itemize}
            Prévu :
                \begin{itemize}
                    \item 5 postes
                    \item 2 imprimantes
                \end{itemize}

        \subsubsection{Département Maintenance}
            61 personnes.

            Existant :
                \begin{itemize}
                    \item 2 postes
                    \item 2 imprimantes
                \end{itemize}
            Prévu :
                \begin{itemize}
                    \item 42 postes (2 pour l'atelier principal, 1 par atelier externe)
                    \item 42 imprimantes (2 pour l'atelier principal, 1 par atelier externe)
                \end{itemize}


        \subsubsection{Département Achat}
            2 personnes.

            Existant :
                \begin{itemize}
                    \item 2 postes
                    \item 2 imprimantes
                \end{itemize}
            Prévu :
                \begin{itemize}
                    \item 2 postes 
                    \item 2 imprimantes
                \end{itemize}

        \subsubsection{Chantiers}

            Existant :
                \begin{itemize}
                    \item 10 postes
                    \item 0 imprimantes
                \end{itemize}
            Prévu :
                \begin{itemize}
                    \item 40 postes
                    \item 40 imprimantes
                    \item n smartphone TODO n combien ?
                \end{itemize}
    
