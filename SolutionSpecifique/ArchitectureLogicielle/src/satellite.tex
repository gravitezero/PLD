\section{Serveur VPN}
    Le serveur VPN sera OpenVPN.

\section{Base de données}
    La base de données devra être répartie et compétitive.
    Notre choix s'est porté sur la base de donnée Oracle pour ses performances et son interfaçabilité.

\section{OSs}
    \subsection{Serveurs}
        Le système d'exploitation des serveurs sera de type UNIX.
        Le choix se portera sur debian pour sa facilité d'utilisation et de maintenance. Suivant la préférence du service informatique déjà en place, un autre système Unix pourra être employé.
    \subsection{Postes clients}
        Le système d'exploitation des postes clients est sans importance étant donné que l'application métier sera portable. En effet, elle sera accessible soit à l'aide d'une application web, soit à l'aide d'un client local multi-plateforme développé avec le framework Qt.

        Le choix par défaut du système d'exploitation des postes clients sera à faire avec les clients. Nous conseillons l'utilisation de Ubuntu Linux pour des raisons de sécurité et de facilité d'utilisation.

        Concernant les postes clients, certaines limitations sont tout de même à garder en tête. Le système devra pouvoir se connecter au réseau VPN. La plupart des smartphone en soit capable, mais ce sera un paramètre déterminant dans le choix du produit finale.
        (Pour exemple, la plupart des smartphone basé sur Android, iOS ou Blackberry OS sont capable de se connecter au réseau VPN.)

        Suivant la politique de sécurité de l'entreprise, nous pouvons envisager deux solutions.
        Si, parmi les employés, certains ont déjà un poste mobile (téléphone ou laptop), il peut être envisageable de ne pas en racheter un en remplacement. Cependant, cela peut être source de fuites d'informations, si des appareils capable de se connecter à la base de données se trouve en dehors de l'enceinte de l'entreprise.
        Il peut donc être envisagé de ne pas permettre ni au smartphone, ni au laptop de sortir de l'enceinte des chantiers ou des locaux de l'entreprise. Cette décision sera à prendre avec l'accord du client.
