
\section {suivi de chantier}

Ce package est principalement destiné à être utilisé directement sur le chantier par des contrôleurs utilisant des PDAs. Les principales contraintes auxquelles est soumis ce module sont : une grande lisibilité des informations sur appareils possédant des écrans de taille réduite, une facilité d'utilisation à l'aide d'écrans tactiles souvent présents sur les appareils mobiles

	\subsection {Objectif}
	Le suivi matériel consiste en la visualisation en temps réel du matériel présent sur site. Ce package sera principalement utilisé par les chefs de chantiers 		afin d'être en mesure d'effectuer la plannification des travaux de la journée en fonction de la présence du matériel requis.

	\subsection {Interface}
	Ce module se présentera dans l'application finale sous la forme d'un tableau permettant de visualiser l'état du matériel présent sur le chantier. Il devra comporter un système de filtrage rapide, simple d'utilisation, afin de sélectionner le matériel par catégorie.



\section {Facturation}


	\subsection {Objectif}
	Le but de ce module est de pouvoir facturer les chantiers pour les matériels utilisés. Il peut être utilisé directement sur site afin de visualiser le matériel commandé ou au siège afin de créer les factures de demande de matériel des différents chantiers.

	\subsection {Interface}
	L'interface de ce module pourra varier en fonction du poste de l'utilisateur : un ouvrier/chef de chantier pourra uniquement visualiser les facture afin de vérifier les dates de commande, le type et la quantité de matériel demandé sur le chantier...) tandis qu'un comptable du siège pourra créer et éditer les factures présentes afin de facturer les différents chantiers en fonction du matériel requis.
