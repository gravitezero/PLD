\section{Stock}
\subsection{Objectif}
Le package stock est destiné à être utilisé pour gérer en temps réel la quantité et le mouvement d'un nombre relativement grand les articles des stocks matériels. L'utilisateur est capable bien sur d'exécuter des commandes basiques pour saisir et documenter des stocks en gérant la réception, la délivrance, les prélèvements de stocks et le conditionnement, les reports de transfert. En outre le ce package permet à l'utilisateur d'optimiser le flux des biens et d'améliorer la rotation de stocks en analysant les historiques des données.\\  

\subsection{Interface}
Le module permet de visualiser les mouvements, la quantité restante...de stocks, d'exporter en plusieurs formats.\\
Le module doit être capable de s'intégrer avec les autres packages pour échanger les données et il doit être aussi facile à utiliser en raison de diversité de niveau de connaissance informatique des utilisateurs.

\section{Planification}
\subsection{Objectif}
Ce package est destiné à être utilisé pour planifier d'affectation et d'assurer l'affectation du matériel aux différents chantiers. Il permet de gérer aussi la réception et l'envoie du matériel et la gestion du parc matériel tout respectant les besoins de l'entreprise et en optimisant le coût de stockage. Il permet aussi  à l'utilisateur d'exécuter les commandes pour traiter et analyser efficacement les demandes d'approvisionnement en provenance des plusieurs chantiers. Il permet aussi de consulter les informations des fournisseurs, historiques de données et d'analyser les statistiques pour optimiser le processus de planification.\\ 
Ce module devrait aussi permet de simplifier et automatiser quelques routines opérationnelles répétitives. Il est aussi personnalisable et reconfigurable pour les futurs besoins. \\
\subsection{Interface}
Ce module permet aussi de visualiser les données, les opérations.\\ 
Facile à apprendre, à utiliser. 
