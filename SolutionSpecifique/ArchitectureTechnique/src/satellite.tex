\section{Introduction préliminaire}

    Dans les améliorations technologique à apporter à la direction matériel, il nous faut, non seulement un moyen de gérer le matériel, mais aussi un moyen de communiquer et d'accéder à cette gestion sur les chantiers.

    Comme les différents sites ne seront pas proches géographiquement, il faudra mettre en place un réseau privé virtuel entre les différents sites, et avec les clients mobiles (smartphone et site chantier).

    Les différents sites et le sièges profiteront d'un accès câblé à l'internet, il n'y auras donc aucune difficulté à mettre en place un réseau privé virtuel.
    En revanche pour les sites de chantiers, l'idéal serait de profiter d'un accès câblé à l'internet, mais ce ne sera pas toujours possible.
    On peut penser alors à un accès par satellite, ou par réseaux 3G/EDGE.
    Pour ces dernier cas, il serait préférable de mettre en place, comme pour les sites, une base de donnée locale afin d'apporter les modifications et de les porter ensuite vers le serveur globales, minimisant ainsi la quantité de données à envoyer.

    Le siège accueillera donc le serveur de données principale, ainsi qu'un serveur de sauvegarde pour éviter toute perte, et pouvoir redémarrer rapidement la production en cas de panne.
    Les différents sites fixes accueillerons également un base de données secondaire, plus petite, permettant uniquement d'enregistrer des modifications en cas d'inaccessibilité au serveur central.

    Pour rester opérationnel sur les chantier non couverts par le réseaux 3G/EDGE, il serait envisageable de mettre en place des relais Wifi faisant la connexion entre le réseaux interne du site de chantier, et les appareils mobiles sur le terrains.

    Pour permettre aux utilisateurs de manipuler cette base de donnée, il faudra leur fournir chacun un poste client, ainsi qu'un réseaux local de connexion, et quelques services d'impressions.


\section{Liste du matériel}
\begin{itemize}
	\item 1 gros serveur d'application
	\item 2 gros serveur de données.
	\item 1 serveur de réseau privé virtuel
    \item 1 firewall
	\item 1 routeur 

	\item 1 serveur d'application / données par site
	\item 1 client de réseau privé virtuel par site
    \item 1 firewall par site
	\item 1 routeur par site

	\item 1 serveur d'application / données mobile par chantier
	\item 1 client de réseau privé virtuel par chanter
	\item 1 routeur par chantier
    \item 1 firewall par chantier
	\item 1 borne relais wifi de secours par chantier
	\item 2 smartphone par chantier

	\item 1 poste client par utilisateur
	\item 1 imprimante pour 20 utilisateurs environ
\end{itemize}

\section{détails suplémentaires}

    Notre solution de communication se basera principalement sur l'utilisation d'un VPN.

    Chaque site accueil déjà un serveur d'application, ce serveur auras à sa charge un serveur VPN fournissant à tout les clients un accès de type intranet identique sur tout les site.
    Les terminaux mobiles auront également accés à ce VPN par le biais de connection mobile, ou par wifi.

    Tous les clients se connecterons donc à ces serveurs à l'aide de clients VPN et profiterons ainsi d'un acces sécurisé à toutes les données necessaires.

    Pour ce qui est du reseaux, l'installation sera principalement impacté par la disposition des locaux.
    Dans tout les cas, on ne pourra se passer, par site, d'un firewall et d'un routeur, suivi d'un ou plusieurs switch permettant de distribuer le reseaux dans chacune des infrastructures.

    Etant donné que la pluspart des employé auront une grande mobilité, il peut être judicieux de les équiper uniquement de postes clients portables.

    \subsection{Siége}

        \subsubsection{Serveur d'application / de données}
            Le siège comporteras la base de données principale. Elle se doit d'être donc la plus performante possible, et résister à la charge.
            Le serveur sera de type rack et sera installé dans une salle dédié réfrigéré avec accès réglementé.
            Cette salle comporteras également, le firewall, le routeur et le switch.

            Le besoin en stoquage et en réactivité ne justifie pas l'utilisation de cluster, ni de base de données disjointe.
            Le choix le plus judicieux est donc un serveur unique performant et disposant d'une grande quantité de stoquage.
            Pour une estimation du prix en vue d'un choix précis, on peut le borner très généreusement à 10 000€.

            Ce serveur sera le coeur de toutes l'installation, autrement dit il sera le plus critique, il est important qu'il soit de la meilleur qualité possible.

            Il permettras l'execution du serveur VPN ainsi que de l'application.

        \subsubsection{Réseau : Firewall, Routeur et switch}
            Afin de connecter les postes clients aux serveur, l'installation d'un réseau local est indispensable.
            Cette connection se fera par l'intermediaire d'un switch sur lequel le serveur et tous les postes clients se connecterons.
            Un routeur en amont permettras notamment la distribution d'adresse IP.

            Une estimation du prix du routeur et du switch pourras être chacun de 1000€.
            
            Afin de connecter ce réseau local à l'exterieur, l'accès internet sera fourni par l'intermédiaire d'un firewall.
            Ce firewall sera de qualité professionnel, et sera aussi important que le serveur lui-même.
    
            Son prix pourras être estimé à 1000€.
        
        \subsubsection{Postes clients}
            TODO compter le nombre de gens

    \subsection{Sites secondaires}

        \subsubsection{Serveur d'application / de données}
            Tout comme le siège chacun des sites secondaire comporterons un serveur hebergeant l'application, et un clone de la base de données du siège.

            Ces sites étant moins important en effectif, la charge sur le serveur sera donc moindre.
            Cependant, on peut s'attendre à une expansion de l'entreprise, il serait donc judicieux d'investir dans du materiel performant, sans être inutilement couteux.
            
            Le choix le plus judicieux sera probablement une tour simple dans une pièce unique réfrigéré avec accès reglementé.
            Comme sur le site principal, cette pièce comporteras le firewall, le routeur et le switch.

        \subsubsection{Réseau : Firewall, Routeur et switch}
            L'infrastucture sera sensiblement la même que sur le site principale, je ne la détaillerais donc pas ici.

   \subsection{Chantier}
