Sommaire
I.       INTRODUCTION.. 3
I.I. Objet du projet 3
I.II. Contexte général du projet 3
II.      LIVRABLES. 3
II.I. Initialisation et Organisation du projet 3
II.II. Expression des besoins. 3
II.III. Construction des scénarios. 3
II.IV. Evaluation des scenarii 3
II.V. Recette. 4
III.           MODE OPERATOIRE ET PHASAGE. 4
III.I. Choix de la méthode. 4
III.II. Phases. 4
III.II.I. Initialisation. 4
III.II.II. Expression des besoins. 4
III.II.III. Analyse et conception des solutions informatiques et   organisationnelles. 4
III.II.IV. Elaboration et évaluation des scénarii 4
IV.          IDENTIFICATION DES ACTIVITES ET DES TACHES. 5
V.      ORGANISAATION DE L’EQUIPE. 5
V.I. Chef de projet 5
V.II. Responsable qualité. 5
V.III. Responsable communication. 5
V.IV. Expert Modelisation/Metier. 5
V.V. Expert ERP. 5
V.VI. Groupe étude. 6
VI.          ANALYSE DES RISQUES. 6
VI.I. Risques. 6
VI.II. Gestion des risques. 6


\section{Introduction}

\subsection{Objet du projet}

Le but de ce projet est d’améliorer le système d’information du domaine \emph{gestion de matériel} de l’entreprise GSTP.

L’objectif de ce projet étant une étude préalable, nous nous limiterons aux phases de spécifications et de conception du système d’information. Nous ne prendrons pas en charge les phases suivants l’étude préalable, c’est à dire : l’étude détaillée, la réalisation..

\subsection{Contexte général du projet}
GSTP est une entreprise de travaux, spécialisée dans le terrassement et le génie civil.
Ceci représente une quarantaine de chantiers, répartis sur un rayon de 500 km.
Au niveau de l’organisation de l’entreprise, sa structure est logiquement divisée en plusieurs services :
\begin{itemize}
    \item Direction Générale (DG)
    \item Direction des ressources humaines (DRH)
    \item Direction des finances et comptabilité (DFC)
    \item Direction informatique (DI)
    \item Direction du matériel (DM)
    \item Direction travaux, études et méthodes (DTEM)
\end{itemize}
La direction des travaux, études et méthodes supervise les chantiers. Chaque chantier est autonome en fonctionnement et financièrement. Ainsi les besoins en matériels sont gérés par la direction des matériels. C’est une relation client-fournisseur interne à l’entreprise.

Nous nous intéresserons plus particulièrement a la direction matériel et ses départements. La DM est attachée à la Direction Générale et a pour missions :
\begin{itemize}
    \item Affecter le matériel au chantier
    \item Assurer la maintenance et la rénovation du matériel
    \item Acquérir de nouveaux matériels
    \item Gérer le stock de pièces de rechange
    \item Louer/Facturer aux chantiers, l’utilisation du matériel
\end{itemize}
Pour gérer l’ensemble de ses départements, la direction matériel utilise de nombreuses applications (obsolètes) de gestion et de planification :
\begin{itemize}
    \item Département Matériel :
        \subitem Gestion de planning
        \subitem Facturation
    \item Département maintenance:
        \subitem Gestion de stocks de pièces de rechange
        \subitem Planification de la maintenance
    \item Département achat :
        \subitem Gestion des fournisseurs
        \subitem Gestion des bons de commande
\end{itemize}


Ces applications sont indépendantes les unes des autres et ne sont intégrées dans aucun système d’information.

\section{Livrables}

Lors de la phase d’étude préalable, des livrables bien définis doivent être fournis aux clients. Ces derniers sont remis lors d’étapes spécifiques présentées ci-après.  L’objet de cette partie est de décrire le rôle et le contenu de chacun des ces livrables.
\subsection{Initialisation et Organisation du projet}

A l’initialisation, deux documents doivent être rédigés. Lors de cette étape, il ne s’agit pas de chercher des solutions informatiques, mais de définir le cadre dans lequel nous nous attacherons à évoluer :

\begin{itemize}
    \item Document d’initialisation : ce document décrit notre démarche pour réaliser le projet, il présente les informations suivantes :
    \begin{itemize}
        \item présentation du contexte global et des objectifs clients.
        \item les livrables attendus
        \item le mode opératoire et le phasage
        \item définition des tâches et planning,
        \item l’organisation de l’équipe
        \item l’analyse des risques
    \end{itemize}
    \item Plan d’assurance qualité : ce document décrit la mise en place de la politique qualité dans le contexte du projet. Il contient :
    \begin{itemize}
        \item la description des documents (dont les livrables) sur le plan de la mise en forme
        \item le cycle de vie des documents
        \item les ressources et outils
        \item les modalités de validations internes et de recette
        \item un annexe (contenant des parties de documents types ou des modèles)
    \end{itemize}
\end{itemize}

\subsection{Expression des besoins

Cette étape doit être réalisée en interaction avec le client. Il s’agit de comprendre ses attentes afin de les reformuler dans un document : le dossier d’expression des besoins.

Il contient :
\begin{itemize}
    \item une présentation du contexte du projet (approche métier),
    \item les éventuelles orientations stratégiques de la MOA
    \item une analyse de l’existant (dont le SI)
    \item la cible fonctionnelle (modèle de référence des activités et processus de l’entreprise).
    \item les écarts avec l’existant (les dysfonctionnements)
    \item les attentes des partenaires
    \item le benchmarking
    \item les thèmes de progrès
\end{itemize}

\subsection{Construction des scénarios}

Un unique document sera fourni. Dans ce rapport, deux scénarios de mise en oeuvre seront envisagés : une solution spécifique et une solution standard de type ERP.

Le document contiendra pour chaque scénario la démarche préconisée :
\begin{itemize}
    \item la nouvelle organisation
    \item l’architecture technique
    \item l’architecture applicative
    \item l’architecture logicielle
\end{itemize}
\subsection{Evaluation des scénarios}

Les deux scénarios présentés dans la partie précédente doivent ensuite être évalués et comparés afin de choisir celui que nous adopterons. Un livrable explicitant notre choix sera fourni, il doit permette au client de comprendre en quoi la solution choisie répond le mieux à son besoin.

Pour chaque scénario, on va rassembler les éléments de choix, à savoir les points forts et les points faibles.


\subsection{Restitution}

C’est la dernière étape, un dossier bilan doit être livré. Le projet est également présenté oralement durant un rendez-vous client.

Durant la présentation finale (powerpoint), nous exposerons notre démarche, présenterons les deux solutions et expliquerons les raisons de notre choix. Il s’agira de convaincre en mettant en avant les points forts de notre projet.

En ce qui concerne le dossier bilan, il vient conclure la phase d’étude préalable. Il souligne :
\begin{itemize}
    \item les évolutions majeures apportées au produit livré par rapport à la définition présentée dans le dossier d’initialisation.
    \item le plan de charges est actualisé, il met en avant les écarts et explique l’origine de ces écarts.
    \item une  synthèse des difficultés rencontrées.
\end{itemize}

Un certain nombre de documents de suivi sont également réalisé tout au long du projet. Il sont cités ici à titre indicatif car il ne s’agit pas de livrables :
\begin{itemize}
    \item Tableau d’avancement des livrables intermédiaires
    \item Tableau de suivi des charges
        \subitem Fiche de suivi individuel par séance
        \subitem Fiche de suivi global par séance
\end{itemize}

\section{Mode opératoire et phasage}

\subsection{Choix de la méthode}

Afin de réaliser cette étude préalable, nous avons opté pour la méthode MERISE, simplement parce qu'elle permet une décomposition du système d'information de l'entreprise en domaines et processus facilement analysables et donc utiles pour notre étude préalable. Nous capitalisons aussi sur le fait que l’existant est basé sur ce modèle.

\subsection{Phases}

Pour Chaque phase, nous préciserons sont but, son déroulement et le(s) livrable(s) attendu(s). Les phases de notre étude préalable sont les suivantes :

\subsubsection{Initialisation}

\subsubsubsection{Buts}
\begin{itemize}
    \item Cibler le champs d’étude du projet
    \item Identifier les contraintes et risques
    \item Elaborer notre démarche
    \item Élaborer un plan d’assurance qualité
\end{itemize}


\subsubsubsection{Déroulement}
\begin{description}
    \item[Contexte général]{
        Il s’agit de faire une introduction présentant brièvement l’entreprise, l’etat du service existant, ainsi que notre rôle dans ce projet.
    }
    \item[Livrables]{
        Il s’agit d’élaborer une liste exhaustive des livrables.
    }
    \item[Mode opératoire et phasage]{
        Il s’agit de choisir les méthodes à adopter, découper le projet en plusieurs phases.
    }
    \item[Activités / taches]{
        Il s’agit d’identifier les taches et activités de chaque phase, les répartir entre les différents collaborateurs selon un planning prévisionnel.
    }
    \item[Organisation de l’équipe]{
        Il s’agit de définir le rôle de chaque membre ainsi que ses principales missions au sein du projet.
    }
    \item[Analyse des risques]{
        Il s’agit de faire une analyse prévisionnelle des risques liés au projet et élaborer un plan de gestion de ces risques.
    }
\end{description}

\subsubsubsection{Livrables}
\begin{itemize}
    \item Dossier d’initialisation
    \item Plan d'Assurance Qualité
\end{itemize}

\subsubsection{Expression des besoins}

But :
\begin{itemize}
    Identifier les thèmes de progrès pour restreindre les futurs scénarios tout en répondant au mieux aux attentes du client.
\end{itemize}


\subsubsubsection{Déroulement}
\begin{description}
    \item[Contexte du projet dans l’entreprise GSTP]{
        comprendre le modèle métier de l'entreprise, identifier les activités , les directions et services concernés par le projet ainsi que les processus stratégiques à analyser.
    }
    \item[Analyse de l’existant]{
        etudier l’existant organisationnel et informatique afin d’identifier les ecarts par rapport à la stratégie de l’entreprise ainsi que les processus à modifier. 
    }

    \item[Normes et benchmarking]{
        cette étape consiste à se renseigner sur les différentes normes et benchmarks existants, mais aussi au près des concurrents afin de comprendre leur méthode, identifier les avantages qu’ils en tirent.  Le but final étant de retenir les “best practice”.
    }

    \item[Cible de référence]{
        il faut élaborer un modèle de référence des processus de l’entreprise à partir des dysfonctionnements relevés, des “best practice” retenue, des attentes clients,..  et indépendemment  des moyens organisationnels et informatiques.
    }

    \item[Thèmes de progès]{
        identifier les axes d’amélioration.
    }
\end{description}

\subsubsubsection{Livrables}
\begin{itemize}
    \item Dossier d’expression des besoins (EB)
\end{itemize}

\subsubsection{Analyse et conception des solutions informatiques et organisationnelles}

\subsubsubsection{Buts}
\begin{itemize}
    \item[Proposer deux solutions distinctes]{
        l’une étant spécifique (construite de A à Z, pour répondre le plus précisément possible aux besoins), l’autre plus standard (basé sur des systèmes standards de type ERP, qui seront adaptés au besoin).
    }
\end{itemize}


\subsubsubsection{Déroulement}
\begin{itemize}
    \item Analyse et conception
        \subitem Définitions des stratégies d’informations
        \subitem Analyse générale de l’architecture applicative cible
        \subitem Conception générale de l’architecture applicative cible
    \item Démarche pour la mise en place d’une solution spécifique
        \subitem Analyse des impacts organisationnels
        \subitem Analyse des impacts informatiques (architectures technique, applicative, logicielle).
    \item Analyse  des solutions existantes du marché/Choix d’une solution :
        \subitem Analyse des impacts organisationnels
        \subitem Analyse des impacts informatiques (architectures technique, applicative, logicielle).
\end{itemize}

\subsubsubsection{Livrables}
\begin{itemize}
    \item Dossier Description des Scénarios
\end{itemize}

\subsubsection{Évaluation des scénarios}

\subsubsubsection{Buts}
\begin{itemize}
    Évaluer les différents scénarios et en faire ressortir les avantages et inconvénients de chacun afin de construire une étude comparative..
    Choisir la solution qui nous semble la plus adaptée aux besoins du clients.
\end{itemize}

\subsubsubsection{Déroulement}
\begin{itemize}
    \item Évaluation des solutions : il s’agit de comparer les deux scénarios et d’en voir les différences. Une présentation des avantages/inconvénients de ces scénarios semble intéressante à produire.
    \item Choix \HUGE{TODO}
\end{itemize}

\subsubsubsection{Livrables}
\begin{itemize}
    \item Dossier de choix
\end{itemize}

\section{Identification des activités et des tâches}

Pour planifier l’étude préalable, il est fondamentale d’identifier l’ensemble des activités et les tâches qui y sont associées. Celles que nous avons identifié sont :
\begin{itemize}
    \item Suivi de projet : \bf{A1}
        \subitem Organisation
        \subitem Planification
        \subitem Évaluation
        \subitem Pilotage/Suivi

    \item Gestion/Contrôle de documents : \bf{A2}
        \subitem Diffusion PAQ
        \subitem Respect PAQ
        \subitem Validation des livrables
        \subitem Organisation des réunions internes
        \subitem Organisation des rencontres avec le client

    \item Production : \bf{A3}
        \subitem Elaboration des livrables
        \subitem Réalisation des rapports
        \subitem Mise en commun des informations
\end{itemize}

Pour representer l’ordonnancement des tâches, nous allons utilisé un diagramme de GANT (fait avec MS Project) qui montrera le positionnement des tâches sur l’échelle du temps et l’utilisation des ressources (membres du projet).


Elaboration du GANT du projet PLD

    Taches de l’activité A1 (Suivi de projet)
        Organisation le projet
        Elaboration du planning prévisionnel
        Prise en main de MS Project
        Elaboration du diagramme de GANTT du projet
        Supervision du travail des collaborateurs
        Redaction des fiches de suivi
            Fiche de suivi d’avancement des livrables intermédiaires
            Fiche de suivi individuel
            Fiche de suivi global


    Taches de l’activité A2 (Gestion/Contrôle de documents)

        Diffuser le  PAQ à tous les collaborateurs
        Controler la qualité des documents (mise en forme et cohérence des documents) par le RQ et le CP
            Contrôle du document d’initialisation
            Contrôle du Plan d’assurance qualité
            Contrôle du dossier d’expression des besoins
            Contrôle du dossier de description des scénarios

            Contrôle du dossier modélisation et configuration standard
            Contrôle du dossier de choix

        Validation des livrables (relecture) par le RQ et le CP
            Validation du document d’initialisation
            Validation du Plan d’assurance qualité
            Validation du dossier d’expression des besoins
            Validation du dossier de description des scénarios

            Validation du dossier modélisation et configuration standard
            Validation du dossier de choix

        Organiser de façon périodique des réunions internes
            Revues ponctuelles en cas de questions en suspens
            Réunion de chantier en début de séance
            Réunion de coordination en fin de séance
        Organisation des réunions avec le client
            Réunion ponctuelle sur demande du responsable communication



    Taches de l’activité A3 (Production )
        Phase 1
            Lecture du sujet
            Analyse du champ d’étude du projet
            Ebauche du dossier d’initialisation
            Ebauche du PAQ
            Redaction du dossier d’initialisation
                Objet et Contexte du projet
                Livrables attendus
                Mode opératoire et phasage
                Identification des taches et activités
                Organisation de l’équipe
                Analyse des risques
            Rédaction du paq
                Description des documents sur le plan de la mise en forme
                Modalité de validations internes et recette
                Cycle de vie des documents
                Ressources et outils
        Phase 2
            Analyse du SI existant
                SI Organisationnel
                    Analyse les modèles existants (MCD, MCT..)
                    Mise à jour des modèles existants (MCD, MCT..)
                SI Informatique
                    Analyser l’existant informatique
                    Synthétiser et rediger l’existant informatique
            Normes et benchmarking
                Collecter des informations pour le benchmarking : les “best practice”
                Synthetiser et rediger la partie sur le benchmarking
                Comparer avec l’existant et detecter les dysfonctionnements
                Rediger les dysfonctionnements
            Cible de fonctionnement
                Etudier les attentes du client, les informations collectées et les dysfonctionnements
                Rediger les attentes du client
            Thèmes de progrès
                Rédiger les thèmes de progrès stratégiques
                Rédiger les thèmes de progrès fonctionnelles
                Rédiger les thèmes de progrès technologiques
        Phase 3
            Elaboration de la solution spécifique : S1
                Etudier la nouvelle organisation
                Etudier l’architecture technique
                Etudier l’architecture logicielle
                Etudier l’architecture applicative
                Rediger le scénario de S1
            Elaboration de la solution standard type ERP : S2
                Etudier les solutions standards du marché
                Choisir une solution du marché
                Etudier la nouvelle organisation
                Etudier l’architecture technique
                Etudier l’architecture logicielle
                Etudier l’architecture applicative
                Rediger le scénario de S2
        Phase 4
            Dossier de Choix d’un scénario
                Evaluer et comparer les scénarios
                Evaluer les risques
                Evaluer les couts
            Dossier Bilan
                Evaluer l’évolution des livrables
                Etablir le bilan des charges
                Synthèse des difficultés
            Présentation orale
                Rédiger la présentation orale
                Simuler la présentation orale



\section{Organisation de l'équipe}

L’équipe projet sera organisé ainsi :
\subsection{Chef de projet : Naby Daouda \textsc{Diakite}}

Son rôle sera :

    Suivi stratégique du projet
        Evaluation risques
        Respect des objectifs
        Respects des délais
    Pilotage opérationnel
        Planification des taches
        Suivi et encadrement des tâches
    Organisation humaine
        Définition du role des membres et leur responsabilité
        Résolution de conflits et arbitrage
    Pilotage de la production
        Suivi des résultats et livrables
        Méthodes et outils

    Production des livrables

\subsection{Responsable qualité : Etienne \textsc{Brodu}}

Son rôle sera :

    Charte qualité
    Cohérence entre les livrables
    Rédaction, MAJ  et Respect PAQ
    Production des livrables


\subsection{Responsable communication : Johann \textsc{Chazelle}}

Son rôle sera :

    Communication interne et externe
    CR de réunions
    Recensement des difficultés des collaborateurs
    Production des livrables



\subsection{Expert Modelisation/Metier : Chafik \textsc{Bachatene}}

Son rôle sera :

    Identification des processus métiers
    Analyse et Conception de l’architecture générale
    Aide à la Modélisation et configuration des solutions 


\subsection{Expert ERP : Baptiste \textsc{Lecornu}}

Son rôle sera :

    Connaissance des normes et benchmarks ERP

    Aide à la Modélisation et configuration des solutions 


\subsection{Groupe étude : Adrien \textsc{Brochot} et Thanh \textsc{Phan Duc}}

Son rôle sera :

    Production livrable
    Aide à la planification du projet
    Responsable de la qualité des taches qu’il réalise



\section{Analyse des risques}
\subsection{Risques}

Les risques sont les suivants :

    R1 :  Risque humains (liés aux compétence, abscence, maladie..)
    R2 : Apparition de tâches supplémentaires liées à la saisie des livrables ( rapport)
    R3 : Difficulté d’évaluation du temps nécessaire à chaque tâche(prise en main des outils et méthodes utilisés,...)
    R4 : Spécification incomplète des points à traiter
    R5 : Risque de surqualité
    R6 : Délais tendus
    R7 : Demande régulière de modification durant l’élaboration des solutions



\subsection{Gestion des risques}

Les solutions que nous préconisons sont :

    S1:
        Imposer un certain nombre de règles à suivre pour le bon déroulement du projet et veiller au respect de ceux-ci. Si nécessaire formaliser ces règles sous forme de “règlement intérieur”
        Motiver suffisamment les membres de l’équipe  et répartir les tâches en fonction des profils et des compétences de chacun
        Redistribuer le travail du membre indisponible aux autres membres de l’équipe durant toute la durée de son indisponibilité.
    S2:
    Prévoir des créneaux horaires (hors séance) pour la prise en main des outils utilisés  et la centralisation de façon efficace des différents livrables.

    S3:
    Contrôle du planning prévisionnel et mise à jour de celui-ci et si nécessaire réaffectation des tâches
    S4:
    S’adresser au client pour éclaircir les points flous
    S5:
    Contrôler de façon permanente l’avancement des tâches et les documents produit


    Maquetage

    S6 :


    Planification détaillée du projet avec un GANTT
    Suivi de l’avancement des livrables

    S7 :


    Seuil d’acceptation des modifications
    Report des modifications en fin de projet
    Gestion de versions
















