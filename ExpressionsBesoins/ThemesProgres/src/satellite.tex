\section{Thèmes de progrès stratégiques}
        \begin{itemize}
                \item Limiter le nombre de fournisseurs afin d'établir des relations privilégiés avec ces derniers et pour bénéficier de l'effet de fidélisation qui entrainerait une baise des coûts d'approvisionnement. Cela permettra également aux fournisseurs de réduire les délais de livraison. En plus en établissant une relation privilégié avec ces fournisseurs, ils pourront disposer d'historiques réguliers sur nos commandes et ainsi pourront prévoir des stocks pour couvrir nos commandes.
                \item Informatiser le département matériel, en améliorant les ressources matérielles ainsi que celles applicatives. Et en établissant des communications entre les différents services de la DM et également avec les chantiers.
                \item Planifier l'acquisition du nouveau matériel en se basant sur les historiques et des statistiques qui seront effectuées régulièrement pour mieux prévoir les besoins.
                \item Créer 5 petits sites positionnés stratégiquement dans les régions (dont un au siège) ou nous avons le plus de chantiers, afin d'être plus proche des chantiers et de réaliser de réels économies logistiques. Sur ces sites, il y aura : un dépôt de gros matériel, un magasin de pièces de rechange et un atelier de maintenance.
                \item Voter un budget annuel pour la DM pour la rendre plus indépendante financièrement (cela implique quel prépare un dossier ou elle explique ces dépenses prévisionnels).
                \item Définir des indicateurs pertinents de suivi des chantiers ainsi que de gestion du parc matériel, afin d'être le plus performant possible.
                \item Ouvrir un service location au sein de la DM sur le siège qui sera chargera de vérifier les demandes de nos chantiers ainsi que celles de chantiers externes, en cas de risque d'immobilisation long de matériels. Elle demandera aux sites ayant à disposition les matériels concernés de les prêter aux chantiers externes ayant un besoin pour ce matériel. Cependant, nos chantiers restent prioritaires. Ce service location permettra de changer plus régulièrement notre parc matériel, mais la DM n'achètera pas de matériel uniquement dans le but de les louer.
        \end{itemize}

\section{Thèmes de progrès fonctionnels}
        \subsection{Processus d'Achat}
        \begin{itemize}
                \item Centraliser les demandes appartenant au département Achat au sein de la nouvelle application. 
				Chaque magasinier des 5 sites pourra saisir ces commandes sur cette application, il pourra également les suivre.
                \item Limiter le nombre de fournisseur et optimiser les transports (localisation prise en compte dans le choix)
				\item Utiliser des outils d'aide à la décision agrégeant les données remontées des services maintenance et 
				du service location pour cibler les achats.
        \end{itemize}

        \subsection{Processus de Planification}
        \begin{itemize}
                \item Établir un historique des besoins en pièces de rechange afin de définir un planning prévisionnel des besoins. Les coûts de stock et le temps de livraison des fournisseurs sont ainsi limités.
                \item Effectuer des statistiques sur les pannes les plus régulières afin de prévoir des maintenances préventives plus efficacement.
                \item Effectuer des statistiques également sur les achats afin de mieux les préparer. Cela permettre également au département de faire en début d'année une demande de budget plus précis à la direction générale.
        \end{itemize}

        \subsection{Processus d'Affectation/Restitution}
        \begin{itemize}
                \item Modifier en temps réel l'état de disponibilité d'un matériel, au moment ou il est affecté à un chantier ou au moment ou il est restitué par le chantier. Cela peut avoir des avantages, permettre par exemple d'envoyer directement un matériel d'un chantier vers un autre chantier, ou juste de passer au site de la région pour y effectuer une rapide maintenance préventive.
                \item Etablir des fiches de suivis des machines (avec leur état sur chantier, en maintenance, ...).
        \end{itemize}

        \subsection{Processus de Facturation}
        \begin{itemize}
                \item Collecter en temps réel les informations sur le taux d'exploitation du matériel par les chantiers, les heures de main d'oeuvre, les coûts en pièces de rechange permet d'effectuer dans le temps le processus de facturation et d'éviter les grandes montées de charge en fin de mois.
        \end{itemize}

        \subsection{Processus de Maintenance}
        \begin{itemize}
                \item Etablir des types de panne pour les opérations de maintenance afin d'accélérer les réparations et de réduire le temps d'immobilisation du matériel durant ces opérations.
                \item Etablir des statistiques d'exploitation des machines durant les opérations de maintenance (curatives et préventives) afin de mieux les exploiter et réduire les coûts liés au matériel.
                \item Collecter pour les nouvelles études statistiques des données en élaborant un relevé d'exploitation du matériel, de la main d'œuvre, de la consommation des pièces de rechange chaque semaine au lieu de chaque mois.
                \item Gérer au niveau de la nouvelle application toute la maintenance, en permettant aux chefs de chantiers de signalés des pannes, des opérations urgentes de maintenance. Cela permettra également au service maintenance d'être plus efficace en planifiant mieux les opérations de maintenance.
                \item Établir une fiche électronique d'intervention pour chaque opération de maintenance. Ces fiches seront accessibles à tous les services de la DM pour mieux faire leurs choix.
        \end{itemize}

        \subsection{Processus de communication interne}
        \begin{itemize}
                \item Mettre en place une homogénéité au sein des applications, afin d'avoir une cohérence des données et de réduire les redondances de collecte d'informations.
                \item Améliorer les communications entre les chantiers et la DM, en permettant aux chefs de chantiers de saisir les informations concernant la disponibilité d'un matériel, une nouvelle demande de matériel, une demande d'intervention urgente. Ce qui permettra d'optimiser l'utilisation du matériel mais aussi de mieux planifier les affectations de matériel aux chantiers, la maintenance, les achats de nouveaux matériels. Bref ce suivi temps réel des chantiers à partir du siège aura d'énormes avantages.
        \end{itemize}
		
		\subsection{Processus de Location}
        \begin{itemize}
                \item Collecter des informations sur l'exploitation interne (chantiers) ou externe(location) des machines.
                \item Etudier la périodicité des maintenance de telle ou telle machine.
				\item Louer le matériel inutilisé et pour lequel aucune maitenance n'est prévue.
        \end{itemize}
		
\section{Thèmes de progrès organisationnels}

La société GSTP est centralisée autour d’un seul site et contrôle en moyenne une quarantaine de chantiers. 
L’une des évolutions envisagée afin de limiter les coûts de transport de matériel est de créer des sites régionaux.
Ainsi, quatre nouveaux sites placés sous la direction du siège sont implantés stratégiquement en France. 
Après une étude menée en partenariat par nos experts et ceux de GSTP, nous pouvons définir la meilleure implantation pour chacun des sous-sites.


Chaque antenne dispose de son propre matériel et de ses pièces de rechanges pour fournir les chantiers de sa région.
De plus, chaque sous site dispose d’une certaine autonomie avec ses propres services de gestion de matériel, de maintenance (curative et préventive) et de secrétariat. 


En ce qui concerne les services maintenance, toutes les ressources ne sont pas conservés. En effet, le département
utilise actuellement un effectif particulièrement important et parfois mal positionné. 
La gestion des achats reste centralisée au niveau du siège mais des demandes peuvent émerger au niveau des sites régionaux (magasiniers des service gestion de matériel).


Enfin, un service location est ajouté au niveau du site central. Il crée une source de revenu supplémentaire et limite l’immobilisation des machines en les louant à des chantiers externes. Cependant, nous choisissons de privilégier nos chantiers, ainsi, la maintenance préventive planifiée est prioritaire à la location du matériel. A terme, les bénéfices de la mise en place d’un tel service peuvent permettre d’assurer un renouvellement plus fréquent du matériel et donc de réduire les coûts de maintenance.


En conclusion, les aménagements suivants sont apportés :
        \begin{itemize}
                \item Changement de la structuration actuelle, en ouvrant 4 nouveaux sites.
                \item Mise à la disposition de la RH des membres du service maintenance non réaffectés.
                \item Recherche de ressource à destination du service location.
                \item Trouver des personnes compétentes pour la mise en place des études statistiques. (???)
        \end{itemize} 

		

\section{Thèmes de progrès technologiques}

Dans les améliorations technologique à apporter à la direction matériel, il nous faut, non seulement un moyen de gérer le matériel, mais aussi un moyen de communiquer et d'accéder à distance à cette gestion.


Comme les différents sites ne seront pas proches géographiquement, il faudra mettre en place un réseau privé virtuel entre les différents sites, les chantiers et avec les clients mobiles (smartphone sur les chantiers), aussi il faudra mettre en place système qui permettra de centraliser les communications et d'assurer un suivi temps réel des chantiers.


Une liste du matériel nécessaire à cela sera fournit durant la mise en place de l'architecture technique.


Ces changements ont pour but :
        \begin{itemize}
                \item Créer un système général permettant de gérer et suivre en temps réel l'activité des chantiers.
                \item Centraliser les communications avec les différentes applications actuelles de la DM.
                \item Sécuriser l'accès aux données de la société
                \item Réduire les redondances d'informations et apporter une certaine homogénéité.
                \item Garder des historiques des différentes actions et mettre en place des plans de reprise efficace en cas de dysfonctionnement.
        \end{itemize}


