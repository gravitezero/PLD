\section{Thèmes de progrès stratégiques}
        \begin{itemize}
                \item Limiter le nombre de fournisseurs afin d'établir des relations privilégiés avec ces derniers et pour bénéficier de l'effet de fidélisation qui entrainerait une baise des coûts d'approvisionnement. Cela permettra également aux fournisseurs de réduire les délais de livraison. En plus en établissant une relation privilégié avec ces fournisseurs, ils pourront disposer d'historiques réguliers sur nos commandes et ainsi pourront prévoir des stocks pour couvrir nos commandes.
                \item Informatiser le département matériel, en améliorant les ressources matérielles ainsi que celles applicatives. Et en établissant des communications entre les différents services de la DM et également avec les chantiers.
                \item Planifier l'acquisition du nouveau matériel en se basant sur les historiques et des statistiques qui seront effectuées régulièrement pour mieux prévoir les besoins.
                \item Créer 5 petits sites positionnés stratégiquement dans les régions ou nous avons le plus de chantiers, afin d'être plus proche des chantiers et de réaliser de réels économies logistiques. Sur ces sites, il y aura : un dépôt de gros matériel, un magasin de pièces de rechange et un atelier de maintenance.
                \item Définir des indicateurs pertinents de suivi des chantiers ainsi que de gestion du parc matériel, afin d'être le plus performant possible.
                \item Ouvrir un service location au sein de la DM sur le siège qui sera chargera de vérifier les demandes de nos chantiers ainsi que celles de chantiers externes, en cas de risque d'immobilisation long de matériels. Elle demandera aux sites ayant à disposition les matériels concernés de les prêter aux chantiers externes ayant un besoin pour ce matériel. Cependant, nos chantiers restent prioritaires. Ce service location permettra de changer plus régulièrement notre parc matériel, mais la DM n'achètera pas de matériel uniquement dans le but de les louer.
                \item 
        \end{itemize}

\section{Thèmes de progrès fonctionnels}
        \subsection{Processus d'Achat}
        \begin{itemize}
                \item 
        \end{itemize}

        \subsection{Processus de Planification}
        \begin{itemize}
                \item 
        \end{itemize}

        \subsection{Processus d'Affectation/Restitution}
        \begin{itemize}
                \item 
        \end{itemize}

        \subsection{Processus de Facturation}
        \begin{itemize}
                \item 
        \end{itemize}

        \subsection{Processus de Maintenance}
        \begin{itemize}
                \item 
        \end{itemize}

        \subsection{Processus de communication cnterne}
        \begin{itemize}
                \item 
        \end{itemize}

\section{Thèmes de progrès organisationnels}
        \begin{itemize}
                \item 
        \end{itemize}

\section{Thèmes de progrès technologiques}
        \begin{itemize}
                \item 
        \end{itemize}


