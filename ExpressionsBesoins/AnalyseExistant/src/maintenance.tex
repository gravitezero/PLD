\section{Métier Maintenance}
\subsection{Services}
Dans l'organisation du département Maintenance, on trouve 2 services:
\begin{itemize}
    \item Service de Gestion des pièces de rechange
    \item Service de Maintenance
\end{itemize}

\subsection{Procédure Maintenace}

Le processus de maintenance comprend :
\begin{itemize}
    \item Effectuer les opérations de maintenance urgentes (depuis de demandes des chantiers)
    \item Procéder au remplacement d'un matériel en panne 
    \item Réaliser la planification de maintenance préventive(selon le planning)
\end{itemize}

Au niveau d'organisation: c'est le service Gestion du matériel qui réalise le planning de maintenance, tandis que le chantier envoie une demande d'intervention à la maintenance dans le cas d'une panne et réclame le remplacement provisoire au service Gestion du Parc Matériel. 


Déroulement: Lors d'une demande de maintenance planifiée ou  depuis un chantier, on procède en identifiant les opérations à effectuer. Ou lors d'une demande d'intervention urgente depuis un chantier, on diagnostique la panne, dans le cas de nécessité, on réalise une demande de matériel urgente.


Suivant la disponibilité du personnel, on affecte l'opération. Si l'opération nécessite des pièces de rechange, on signale pour les obtenir.


A la fin de l'opération on signale au chantier ou au parc matériel que l'objet de la maintenance est à nouveau disponible (et en état de marche...).



\subsection{Etat du système informatique}
\begin{itemize}
    \item Aspect matériel: Le département maintenance est doté de 2 postes et 2 imprimantes, connectées aux réseaux locals.
    \item Aspect logicielle: Il dispose de 2 logiciels 
    \begin{enumerate}
        \item Gestion de stocks et pièces de réchange
        \item Planification de maintenance
    \end{enumerate}
\end{itemize}
