\section{Normes et Benchmarking}
		
		Le benchmarking consiste à situer GSTP par rapport à la concurrence. L'objet de ce document est ainsi de repérer les leaders du marchés et de s'inspirer de leur \emph{best practices}, de leurs fonctionnements et de leurs expériences afin que les pratiques en interne se rapprochent de plus en plus de la  \emph{perfection}.
Dans un premier temps, nous nous attacherons à détailler les entreprises phares du secteur des travaux publics, puis nous nous concentrerons plus spécifiquement aux systèmes informatiques du marché ainsi qu'à ceux mis en place par la concurrence.

		\subsection{Les concurrents et leaders du marché}
		GSTP est une entreprise de travaux, spécialisée dans le terrassement et le génie civil. Elle s'inscrit donc dans le domaine des BTP où l'on peut citer quelques grands groupes : 
				\begin{description}
				    \item[China Railway Construction] (Chine ; CA de 41,3 milliards d'euros)
				    \item[Vinci] (France) avec en particulier Vinci Construction qui est le premier contributeur au chiffre d'affaire du groupe (CA de 31,9 milliards d'euros)
				    \item[Bouygues] (France ; CA de 31,4 milliards d'euros)
				    \item[ACS] (Espagne ; CA de 15,6 milliards d'euros)
				\end{description}

		Nous allons étudier ici les best practices de deux de ces groupes : Vinci et China Railway Construction en s'intéressant tout particulièrement aux systèmes d'informations. Enfin, nous terminerons par énoncer quelques bonnes idées vu chez les PME concurrentes.

						\subsubsection{Vinci Construction}
						Vinci Construction est leader en France et troisième groupe mondial de la construction.
Il réunit un ensemble de compétences dans les métiers du bâtiment, du génie civil, des travaux hydrauliques et des services.
L'un des points fort du groupe réside en sa capacité à s'attacher les services d'entreprises locales pour déployer ses solutions. Ainsi, ces dernières sont à la fois globales et modulables.
D'autre part, On est très loin des chantiers de PME de la construction. 
En effet, le compagnon (terme employé par les chefs de chantier pour parler de leurs ouvriers) ne réalise pas une multitude de tâches, sa mission et son champ d'intervention sur le chantier sont très précis, cela pour des raisons de sécurité, de fiabilité et d'efficacité.

								\paragraph{Modernisation de son Système d'information}
								Le groupe a choisi SAGE pour la mise en place de son FRP (Finance Ressource Planning) :
Les principales fonctionnalités qui ont fait la différence sont le reporting (à destination de la maison mère), la dématérialisation des documents, le portail utilisateur, la Business Intelligence, la gestion full web, et la réponse aux particularités métiers du BTP (facturation à l'avancement, gestion des acomptes, réponse à l'organisation décentralisée du BTP).
Les principaux objectifs de Vinci étaient d'améliorer la productivité des équipes ainsi que la fiabilité et la production des flux financiers.

					\subsubsection{China Railway Construction Corporation}
CRCC est l'un des plus grands groupes Chinois et fait partie des 500 plus grandes firmes mondiales. L'objectif principal du groupe est de devenir l'entreprise de construction la plus compétitive au monde.
Le groupe visant principalement des projets sur le sol chinois commence à exporter son savoir faire à l'étranger en mettant en valeur
sa capacité à réaliser, mais également concevoir d'important projets de construction aux domaines variés (chemin de fer, bâtiment, ...).
Les apports d'une étude poussée de son fonctionnement restent toutefois limités en ce qui concerne GSTP, 
CRCC profitant de l'émergence de la Chine et s'appuyant sur une capacité (financière, humaine, ...) difficilement comparable à celle d'une PME.


							\paragraph{CRCC choisit Inspur ERP spécialisé dans la gestions des firmes}
		Inspur ERP définit le concept de "Headquarter ERP", en effet son système à destination des grands groupes assure une centralisation des décisions.
Un management global peut ainsi être entrepris par CRCC. 
Enfin, Inspur ERP-GS fournit les 10 outils stratégiques de management : gestion financière, gestion des fonds, du budget global, des actifs, de la distribution, des performances ; des outils de business intelligence, d'aide à la décision ; des passerelles d'informations et une gestion des ressources humaines.


		\subsection{Idées et bonne pratiques des PME concurrentes}
				\subsubsection{La location du matériel}
				De nombreuses entreprises de taille comparable à GSTP proposent un service de vente de matériel, mais également de location. 
Ainsi, les machines inutilisées peuvent être louées à d'autres entreprises et les coûts d'immobilisation du matériel s'en trouvent réduits. 
La mise en place d'un tel fonctionnement suppose un choix important : 
privilégier la pleine utilisation du matériel et la maintenance curative au risque de pannes plus fréquentes
ou assurer une maintenance préventive importante et ainsi valoriser le matériel en acceptant qu'il soit parfois immobilisé.

				\subsubsection{Un exemple : Travaux publics bâtiment boulanger}
				TP2B est une PME spécialisée dans les travaux de terrassement, assainissement et génie civil. Avec un effectif de 33 personnes et un Chiffre d'Affaire de 4,2 millions d'euros,
TP2B participe à de nombreux chantiers de l'Est de la France. La particularité de l'entreprise réside en son service location. En effet, elle dispose d'un parc matériel important qu'elle loue à des chantiers externes. D'autre part, pour éviter d'important travaux de maintenance TP2B assure une rotation de remplacement globale ou partielle de ses machines de 2 ans.

				\subsection{une gestion informatisée}
				Actuellement GSTP dispose d'un système d'information restreint. L'outil informatique reste peu utilisé.
A traves l'étude précise des PME concurrentes on constate que de nombreuses sociétés ont choisi d'informatiser leur système notamment les leaders du marché. Certains logiciels utilisés seront détaillés dans la partie suivante relative aux ERP.

				\subsubsection{l'exemple d'ART TP}
				Cette PME d'une trentaine de salarié a choisi d'équipé 8 de ses postes informatique d'un ERP : PROGIB. Bien que n'utilisant pas la totalité des fonctionnalités proposées par le logiciel, ART TP a considérablement amélioré son activité. Par exemple, le suivi des chantiers géré jusqu'alors sur papier voire feuille excel a été considérablement simplifié et amélioré par PROGIB. Finalement, l'informatisation des processus de gestion semble être un moyen efficace pour améliorer son activité.


		\subsection{les principaux ERP du BTP}
		Au fil des années, le marché du Progiciel pour le BTP est devenu très varié, encombré. 
Le développement d'application réclame beaucoup d'investissements tant en matière de spécificités fonctionnelles (budget, devis, comptabilité, finances, décisionnel, facturation, paie) que métier (bordereaux de prix, conception assistée par ordinateur, calcul de structure, des flux, planning, suivi de chantier, ...). 
Résultat : l'offre informatique est très variée. On y trouve de grands généralistes, comme Sage qui partagent la marché avec de nombreuses offres spécialiséEs, comme EBP ou Pro2i et des éditeurs régionaux de type Aquitaine Informatique ou Concept Informatique.
Nous nous intéresserons ici, plus particulièrement aux divers offres génériques de SAP et SAGE ainsi qu'à des logiciels spécifiques en BTP.
				\subsubsection{SAGE X3 Edition : un ERP générique visant un large public}
						Logiciel Full web,
						Objetcifs :
						\begin{itemize}
						    \item Réduire les coûts et les délais
						    \item Améliorer la visibilité sur l'ensemble de vos activités
							\item Optimiser l'interopérabilité de l'ensemble des sites
						    \item Améliorer la satisfaction des clients
							\item Saisir de nouvelles opportunités de business
						\end{itemize}

						\subsubsection{Offre Moyennes et grandes entreprises}
								\url{http://www.sage.fr/mge/logiciels-erp}
								\subsubsection{Offre PME}
								\url{http://www.sage.fr/pme/logiciels-de-gestion/erp}
						Entreprise de plus de 20 ans, Sage a tissé des liens pérennes avec les acteurs stratégiques du bâtiment, les organisations professionnelles, les industriels, 
            les négociants en matériaux, les centres de formation, les experts-comptables, etc. 
Cet ERP propose 2 offres pour le secteur du BTP, l'une s'adressant au PME, l'autre au plus grande entreprise.

L'offre SAGE 100 Multi Devis propose de répondre aux besoins suivants :
\begin{itemize}
  \item Saisir simplement les devis
  \item Réaliser des études de prix complexe
  \item Gérer le déboursé, le prix de revient et maîtriser les marges
  \item Préparer automatiquement les factures et les situations intermédiaires
  \item Gérer les cycles d’achats et la relation avec les fournisseurs
  \item Contrôler la gestion du temps par salariés et par chantier
  \item Assurer la gestion comptable et financière de votre entreprise
  \item Gérer votre personnel en conformité avec les obligations légales\\
\end{itemize}

La section gestion du parc matériel semble peu présente, voire inexistante.
				\subsubsection{SAP}

				SAP ERP est composé d'une centaine de modules fonctionnels bien précis (Material Management, Sales and Distribution,... ). 
Le principal intérêt de SAP ERP est qu'il est totalement flexible. On peut installer tous les modules fonctionnels, ou seulement quelques-uns. 
Aucun superflux. SAP ERP est entièrement paramétrable et s'adapte ainsi aux besoins et à la structure de l'entreprise. 
Grâce à ses fonctionnalités, ce progiciel s'adapte parfaitement au secteur du BTP. 
Enfin, grâce à son environnement de développement, SAP ERP peut être adapté à des besoins spécifiques.\\
\paragraph{Description du module le plus intéressant : Material Management}
Le module MM (Material Management) concerne la gestion des articles d'un point de vue achats et gestion des stocks.
Y sont intégrées des notions telles que :
\begin{itemize}
\item Le calcul des besoins, des réapprovisionnements (MRP - Material Requirements planning)
\item La gestion des achats
\item contrats, demandes d'achats, etc.
\item commandes de biens, de services
\item Mouvements de stocks
\item réceptions de marchandises
\item Valorisation des stocks en intégration avec FI
\item Contrôle des factures
\item Gestion des stocks
\item entrées, sorties, transferts de stocks
\item Gestion des emplacements magasin (WM Warehouse Management)
\item Inventaire\\
\end{itemize}

D'autres modules tel que le module SD (Sales and Distribution) pourront nous intéresser.
SAP ERP peut nous permettre de répondre exactement aux exigences du client. Mais est-il pertinent d'utiliser une telle usine à gaz pour une PME?  

		\subsubsection{PROGIB}
				\url{http://www.quelsoft.com/fiche/progib-m46-43-148.html}
				 
				Progiciel intégré pour les entreprises du bâtiment, des travaux publics et d'espaces verts.\\

Le noyau central est l'analyse en temps réel de la rentabilité et de l'avancement des chantiers et des affaires. A celà s'ajoute des modules complémentaires tels que
 : gestion de stocks avec codes à barres, comptoir de vente, parcs matériels, suivi des contrats de maintenance et petits dépannages, CRM, module décisionnel.
Un des atouts de PROGIB est sa toute dernière nouveauté : le suivi des interventions sur des Pockets-PC et la synchronisation par GPRS.

		\subsubsection{ONAYA}
		ONAYA est un outil de gestion issu de 20 annees de travail du groupe Aquitaine Informatique, en collaboration avec les entreprises de travaux publics. Ce progiciel de gestion integre (ERP/PGI) est lui aussi une application modulaire permettant de gerer et piloter une entreprise de travaux publics. Cette solution gere : les etudes de prix - devis, la facturation, le suivi de chantiers, la
logistique, la saisie nomade, le planning, la comptabilite et la paye. Cette solution couvre quasiment l'ensemble des besoins de gestion et de suivi des chantiers de GSTP (de la DM).

				\paragraph{Gestion des chantiers}
				\begin{itemize}
				    \item Gestion de la nomenclature, production.
				    \item Gestion des achats.
				    \item Gestion des matériels.
				    \item Gestion du personnel.
				\end{itemize}
				
				\paragraph{Gestion des achats}
				\begin{itemize}
				    \item Prise en compte des demandes d'approvisionnement pour un ou plusieurs chantiers.
				    \item Consultation des fournisseurs.
				    \item Etablissement des commandes fournisseurs ou réservation sur stock.
				\end{itemize}
				
				\paragraph{Gestion des stocks}
				\begin{itemize}
				    \item Approvisionnement du stock (commandes et réception des matériaux).
				    \item Approvisionnement des chantiers.
				    \item Statistiques de consommation.
				\end{itemize}
				
				\paragraph{Gestion des matériels}
				\begin{itemize}
				    \item Gestion du parc matériel.
				    \item Gestion de l'atelier.
				    \item Gestion des pièces de rechange.
				\end{itemize}
				
				\paragraph{Gestion du planning}
				\begin{itemize}
				    \item Plan de charges.
				    \item Planning financier.
				    \item Découpage et planification des ressources.
				\end{itemize}
				
		Cette solution est très complète. Certaines parties sont trop détaillées, offrent beaucoup plus de fonctionnalités que ce dont nous avons besoin pour couvrir toute l'activité de GSTP.
			
		\subsubsection{BRZ 7 : ancien Kyetos2}
		
		BRZ 7 est un progiciel intégrée qui permet aux PME BTP d'unifier l'ensemble de leur processus métiers (Etude de prix, Gestion de chantier, gestion financière) autour d'une base de données unique. 
    L'absence de ressaisie apporte des gains de productivité important. 
    Cette solution améliore la performance des utilisateurs en leur offrant des outils d'aide à la gestion: Reporting analytique, 
    planification et logistique chantier, gestion documentaire...\\ 
		BRZ 7 est composé de :
 		
     \begin{itemize}
		  \item \textbf{Pointage Smartphone}
		  \item Gestion commerciale
		  \item Etude de prix et Risk management
		  \item Logistique et planification
		  \item Suivi de chantier
		  \item Gestion des achats et stocks
		  \item Gestion des sous-traitants
		  \item Comptabilité Générale et Analytique
		  \item Gestion des immobilisations
		  \item Contrôle de gestion, états financiers et fiscaux
		  \item Paie et Gestion et ressources humaines\\
		\end{itemize}
		
		BRZ 7 propose aussi plusieurs progiciels séparés. L'un d'entre eux peut nous intéresser. Il s'agit de Phenos.
    Thenos est un outils entièrement personnalisable pour une gestion adaptée à vos besoins : 
    gestion technique, approvisionnement, stock, interventions, planification/logistique, suivi, calcul de rentabilité et intégration comptable. 
		  
		


Nous avons ci-dessous représenter les avantages et les inconvénients des solutions ci-dessus.


\comparatif{SAP ERP}
{
SAP ERP est composé d'une centaine de modules fonctionnels bien précis (Material Management, Sales and Distribution,... ).
Le principal intérêt de SAP ERP est qu'il est totalement flexible. On peut installer tous les modules fonctionnels, ou seulement quelques-uns.
Aucun superflux. SAP ERP est entièrement paramétrable et s'adapte ainsi aux besoins et à la structure de l'entreprise.
Grâce à ses fonctionnalités, ce progiciel s'adapte parfaitement au secteur du BTP.
Enfin, grâce à son environnement de développement, SAP ERP peut être adapté à des besoins spécifiques.
}
{
    \begin{itemize}
        \item Modulable
        \item Evolutif
        \item Adaptable au BTP
        \item Une forte expérience
    \end{itemize}
}
{
    \begin{itemize}
        \item Une usine à gaz
        \item Application lourde
    \end{itemize}
}

 
 
\comparatif{Onaya}
{
Onaya est un ERP spécialement conçu pour les entreprises du bâtiment et des travaux publics (BTP).
Onaya est aussi modulaire mais il ne propose que peu de modules (8).
L'implantation de cet ERP demandera un changement sur l'ensemble de l'entreprise et pas seulement sur la DM.
}
{
    \begin{itemize}
        \item Modulable
        \item Propre au BTP
    \end{itemize}
}
{
    \begin{itemize}
        \item Peu de modules
        \item Remaniement de l'entreprise
    \end{itemize}
}

\comparatif{BRZ 7}
{
Logiciel pour la gestion globale des entreprises de BTP. Constitué de neuf modules : étude de prix, métré, planning, facturation, suivi de chantier, gestion des achats, comptabilité générale et analytique et pointage main d'\oe uvre.
Configuration mono ou multiposte. Possibilité d'installation en réseau (internet ou intranet) pour applications multisites.
}
{
    \begin{itemize}
      \item Pointage Smartphone
      \item Propre au BTP
      \item Multisites
    \end{itemize}
}
{
    \begin{itemize}
        \item Peu connu
        \item Maintenance?             
    \end{itemize}
}

\comparatif{SAGE}
{
  Troisième éditeur mondial de logiciels de gestion, Sage simplifie et automatise la gestion et les processus métier de 6,1 millions d'entreprises dans 70 pays à travers le monde.
  Sage propose une offre complète couvrant les besoins de toutes les entreprises.
  Sage a fait le choix d'une approche décentralisée : chacune de ses 26 filiales dispose d'une autonomie
   de décision et développe localement ses produits afin de répondre avec réactivité aux besoins spécifiques de chaque pays.
   Le développement en France des solutions destinées aux entreprises françaises permet  à  Sage de mettre rapidement à la disposition de ses clients des logiciels conformes à la réglementation locale.
  Cette stratégie permet à Sage de répondre dans des délais très courts  à  l'évolution des besoins des clients : les innovations sont le fruit de l'observation de ces besoins et de l'évolution de la réglementation.
  Sage met rapidement à la disposition des entreprises françaises les outils adaptés à leur croissance.
  Pour s'assurer de l'adéquation de son offre, Sage a mis en place une série d'indicateurs qui sont autant de critères de performance pour ses collaborateurs.
}
{
    \begin{itemize}
        \item International
        \item Développement local
        \item Démarche Qualité
        \item Maintenance
    \end{itemize}
}
{
    \begin{itemize}
        \item Encore une usine à gaz?
        \item Remaniement Complet
    \end{itemize}
}
 
\comparatif{Progib}
{
PROGIB s'appuie sur une équipe et une expérience de 25 ans dans l'informatique de gestion et s'est très fortement spécialisée dans les dernières années.
Aujourd'hui la cible unique de l'entreprise est l'entreprise de bâtiment, travaux publics et espaces verts.
PROGIB traite sa clientèle en direct mais propose également des services de proximité sur l'ensemble du territoire
français par l'intermédiaire de son réseau de distributeurs agréés.
}
{
    \begin{itemize}
        \item Modulable
        \item Propre au BTP
        \item Multisites
        \item Pocket-PC
    \end{itemize}
}
{
    \begin{itemize}
        \item Noyau principal financier
        \item Maintenance?
    \end{itemize}
}
