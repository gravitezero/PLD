\section{Introduction}

Le benchmarking consiste à situer GSTP par rapport à la concurrence. L'objet de ce document est ainsi de repérer les leaders du marchés et de s'inspirer de leur "best practices". 
Dans un premier temps, nous nous attacherons à détailler les entreprises phares du secteur des travaux publics, puis nous nous concentrerons plus spécifiquement sur les systèmes d'informations mis en place par la concurrence.

\section{Les concurrents et leaders du marché}
GSTP est une entreprise de travaux, spécialisée dans le terrassement et le génie civil.
Elle s'inscrit donc dans le domaine des BTP où l'on peut citer quelques grands groupes
\begin{itemize}
    \item Vinci (France) avec en particulier Vinci Construction qui est le premier contributeur au chiffre d'affaire du groupe (CA de 31,9 milliards d'euros)
    \item Bouygues (France ; CA de ?)
    \item China Railway Construction (Chine ; CA de 23,1 milliards d'euros)
	\item ACS (Espagne ; CA de 15,6 milliards d'euros)
\end{itemize}

Nous allons étudier ici les best practices de trois de ces groupes : Vinci, Bouygues et China Railway Construction en s'intéressant tout particulièrement aux systèmes d'informations. Enfin, nous terminerons par énoncer quelques bonnes idées vu chez les PME concurrentes.

\subsection{Vinci Construction}
Vinci Construction est leader en France et troisième groupe mondial de la construction.
Il réunit un ensemble sans équivalent de compétences dans les métiers du bâtiment, du génie civil, des travaux hydrauliques et des services.
L'un des points fort du groupe réside en sa capacité à s'atacher les services d'entreprises locales pour déployer ses solutions. Ainsi, ces dernières sont à la fois globales et modulables.
D'autre part, On est très loin des chantiers de PME de la construction. 
En effet, le compagnon (terme employé par les chefs de chantier pour parler de leurs ouvriers) ne réalise pas une multitude de tâches, sa mission et son champ d'intervention sur le chantier sont très précis, cela pour des raisons de sécurité, de fiabilité et d'efficacité.

\subsubsection{Modernisation de son Système d'information}
Le groupe a choisi SAGE pour la mise en place de son FRP (Finance Ressource Planning) :
Les principales fonctionnalités qui ont fait la différence sont le reporting (à destination de la maison mère), la dématérialisation des documents, le portail utilisateur, la Business Intelligence, la gestion full web, et la réponse aux particularités métiers du BTP (facturation à l'avancement, gestion des acomptes, réponse à l'organisation décentralisée du BTP).
Les principaux objetcifs de Vinci étaient d'améliorer la productivité des équipes ainsi que la fiabilité et la production des flux financiers.

\subsection{Bouygues}
\subsection{China Railway Construction Coorporation}

\subsubsection{CRCC choisit Inspur ERP spécialisé dans la gestions des firmes}
Inspur ERP définit le concept de "Headquarter ERP", en effet son sytème à destination des grands groupes assure une centralisation des décisions et en particulier de la gestion des flux financiers.
Ainsi, China Railway construction peut centraliser les données entrées dans ses différentes entreprises et manager ces dernières à partir du siège.
Enfin, Inspur ERP-GS fournit les 10 outils stratégiques de management : gestion financière, gestion des fonds, du budget global, des actifs, de la distribution, des performances ; des outils de business intelligence, d'aide à la décision ; des passerelles d'informations et une gestion des ressources humaines.


\subsection{La location du matériel}
De nombreuses entreprises de taille comparable à GSTP propose un service de vente de matériel, mais également de location. 
C'est notamment le cas de TP2B qui peut ainsi louer des machines inutilisées à d'autres entreprises (\url{http://www.tp2b.fr/location.htm}).
Cette valorisation des machines inutilisaées permet de limiter les coûts d'immobilisation du matériel et ainsi de créer une source de revenu supplémentaire à l'entreprise.

\section{les principaux ERP du BTP}
Au fil des années, le marché du Progiciel pour le BTP est devenu très varié, encombré. 
Le développement d'application réclame beaucoup d'investissements tant en matière de spécificités fonctionnelles (budget, devis, comptabilité, finances, décisionnel, facturation, paie) que métier (bordereaux de prix, conception assistée par ordinateur, calcul de structure, des flux, planning, suivi de chantier, ...). 
Résultat : l'offre informatique est très variée. On y trouve de grands généralistes, comme Sage qui partagent la marché avec de nombreuses offres spécialiséEs, comme EBP ou Pro2i et des éditeurs régionaux de type Aquitaine Informatique ou Concept Informatique.
Nous nous intéresserons ici, plus particulièrement aux divers offres de SAGE ainsi que logiciels spécifiques PRO2I (PROGIB) et CAAD.
\subsection{SAGE X3 Edition, un ERP générique visant un large public}
Logiciel Full web,
Objetcifs :
\begin{itemize}
    \item Réduire les coûts et les délais
    \item Améliorer la visibilité sur l'ensemble de vos activités
	\item Optimiser l'interopérabilité de l'ensemble des sites
    \item Améliorer la satisfaction des clients
	\item Saisir de nouvelles opportunités de business
\end{itemize}

\subsubsection{Offre Moyennes et grandes entreprises}
\url{http://www.sage.fr/mge/logiciels-erp}
\subsubsection{Offre PME}
\url{http://www.sage.fr/pme/logiciels-de-gestion/erp}

\subsection{Des ERP à destination des PME du BTP}
\subsubsection{PROGIB}
\url{http://www.quelsoft.com/fiche/progib-m46-43-148.html}
 
Progib propose une solution hebergée.
Vue par les clients :
\url{http://www.progib.fr/detail_reference.aspx?id=18} ART TP (30 salariés)
Galliers SAS (152 salariés, 12M euros).

\subsubsection{CAAD}
Logiciel spécialisé en travaux publics. (partenaires de Sage) société conseil AMBS.
A destination des PME :
\begin{itemize}
    \item Module de gestion de sous chantier
    \item Gestion des devis / factures commerciales
    \item Gestion de l'avancement de facturation
\end{itemize}

\url{http://www.caad.fr/caad.htm} pour plus d'info : rubriques téléchargements. Par exemple, fiche chantier : \url{http://www.caad.fr/fiches/pdf/fiche-chantier.pdf}
Sociétés qui utilisent CAAD et pourquoi ?
