\section{M�tier Maintenance}
\subsection{Services}
Dans l'organisation du d�partement Maintenance, on trouve 2 services:
\begin{itemize}
\item Service de Gestion des pi�ces de rechange : 
\item Service de Maintenance :
\end{itemize}

\subsection{Proc�dure Maintenace}

Le processus de maintenance comprend :
\begin{itemize}
\item Effectuer les op�rations de maintenance urgentes (depuis de demandes des chantiers)
\item Proc�der au remplacement d'un mat�riel en panne 
\item R�aliser la planification de maintenance pr�ventive(selon le planning)
\end{itemize}

Au niveau d'organisation: c�est le service Gestion du mat�riel qui r�alise le planning de maintenance, tandis que le chantier envoie une demande d�intervention � la maintenance dans le cas d'une panne et r�clame le remplacement provisoire au service Gestion du Parc Mat�riel. 


D�roulement: Lors d'une demande de maintenance planifi�e ou  depuis un chantier, on proc�de en identifiant les op�rations � effectuer. Ou lors d'une demande d'intervention urgente depuis un chantier, on diagnostique la panne, dans le cas de n�cessit�, on r�alise une demande de mat�riel urgente.


Suivant la disponibilit� du personnel, on affecte l'op�ration. Si l'op�ration n�cessite des pi�ces de rechange, on signale pour les obtenir.


A la fin de l'op�ration on signale au chantier ou au parc mat�riel que l'objet de la maintenance est � nouveau disponible (et en �tat de marche�).



\subsection{Etat du syst�me informatique}
\begin{itemize}
\item Aspect mat�riel: Le d�partement maintenance est dot� de 2 postes et 2 imprimantes, connect�es aux r�seaux locals.
\item Aspect logicielle: Il dispose de 2 logiciels 
\begin{enumerate}
\item Gestion de stocks et pi�ces de r�change
\item Planification de maintenance

\end{enumerate}

\end{itemize}
