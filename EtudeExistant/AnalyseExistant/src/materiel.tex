\section{Métier matériel}
\subsection{Services}
	Dans l’organisation de GSTP, nous pouvons distinguer plusieurs services appartenant au département matériel :
	\begin{description}
		\item[Service Gestion de matériel]~\\
			Il permet de gérer le planning d’affectation du matériel ainsi que son suivi sur les différents chantiers.
		\item[Service Gestion du parc Matériel]~\\ 
			Son rôle consiste à envoyer le matériel disponible sur les différents chantiers.
		\item[Service Facturation Matériel]~\\
			Ce service permet de gérer la facturation en interne du matériel sur les différents chantiers.
	\end{description}	
	Le département matériel permet donc à la fois de suivre l’affectation du matériel sur site mais également de le facturer en interne.

\subsection{Gestion du parc Matériel}
	Le département matériel doit gérer un parc relativement grand de matériels (environ 2.000 au total) composé aussi bien d’engins de travaux, que de camions de transport ou de petits matériels tels que des postes à souder.
Afin de faciliter la gestion de ce parc, chaque matériel est référencé de la manière suivante : il est repéré par un identifiant (code parc) et plusieurs autres valeurs (code type, code groupe, date de mise en service, nombre d’heures de fonctionnement…) permettant de l’identifier.

\subsection{Procédures de gestion du matériel}
	\subsubsection{Procedure de plannification d’affectation du matériel}
	Cette procédure se décompose en deux parties : Il y a tout d’abord la programmation de l’utilisation de matériel : Cette opération permet de prévoir les matériels nécessaires pour les différents chantiers. Elle est mise à jour périodiquement afin de suivre toute modification de besoin de matériel sur les chantiers.
La deuxième partie correspond à la planification de l’affectation de matériels aux chantiers : Elle s’occupe d’envoyer aux services achat et maintenance les demandes de matériel par chantier : demande de location ou d’achat de nouveau matériel et planification des opérations de maintenance du matériel sur site.

	\subsubsection{Procédure de Facturation du matériel pour un chantier}
Cette procédure permet de facturer en interne le matériel sur les différents chantiers. Cette tâche n’est pas assurée par le service Achat car celui-ci ne s’occupe que des factures extérieures.
Cette procédure est périodiquement déclenchée et réalise en premier lieu deux opérations : le calcul de la valorisation du matériel sur les chantiers et le calcul du prix moyen des pièces de rechange utilisées lors des opérations de maintenance. Cette deuxième partie est ensuite précisée afin d’obtenir un calcul précis des coûts de maintenance.
Ces opérations permettent d’effectuer le calcul de Facture matériel pour un chantier.

		
	\subsubsection{Procédure de d’affectation et de restitution du matériel}
Cette procédure gère la présence de matériel sur site depuis sa réception auprès du fournisseur jusqu’à s on retour du chantier.
Lors de la réception  de matériel d’un fournisseur, un avis de livraison est édité. Le matériel est alors envoyé sur le chantier concerné et mis à disposition. Lors de l’affectation du matériel, un étude de matériel manquant est effectuée et une demande de location urgente peut être formulée. Lorsque le matériel n’est plus nécessaire sur le chantier, un avis de restitution de matériel est envoyé et le matériel est alors renvoyé (soit au fournisseur, soit au parc matériel) ou encore envoyé au département maintenance pour une opération de maintenance préventive.

\subsection{Etat du système informatique}
Actuellement, le système informatique comprend une application de gestion du planning ainsi qu’une application de facturation.
Le département matériel est constitué de 3 PCs et de deux imprimantes.

%\subsection{Dysfonctionnements}
%	\begin{itemize}
%	\item Mauvaise planification des affectations
%	\item Immobilisation du matériel
%	\end{itemize}
