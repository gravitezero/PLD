\section{Métier Achat}


\subsection{Services :}
On distingue dans l'organisation de GSTP un ensemble de services participant aux activités d'achat :
\begin{itemize}

   \item Service de gestion des fournisseurs: \\


C'est le service qui traite avec les fournisseurs, étudie les marchés afin d'acquérir des matériels de qualités à des prix intéressants.

    \item Service achat matériel:  \\


C'est le service qui s'occupe des achats des nouveaux matériels hors pièces de rechange.

    \item Service achat pièces de rechange: \\


C'est le service qui s'occupe de l'achat des pièces de rechange

    \item Service location matériel: \\


C'est le service qui s'occupe de la location des matériels lorsque ceux-ci ne sont pa disponibles.
\end{itemize}

GSTP gère donc ses approvisionnements sous deux modes différents : \\
l'acquisition de nouveaux matériels et pièces de rechange et la location de certains matériels.
(On voit que le service qui gère l'achat des pièces de rechanges est différent de celui qui gère l'achat des matériels.)

Cette gestion est une conséquence directe du processus de planification afin de mieux gérer le processus d'approvisionnement.

\subsection{Gestion financière :}
Il existe un budget relatif au renouvellement du matériel, celui-ci est demandé par la DM et validé par la DG. \\
Le département achat commande environ 600 petits matériels par an et fait 2 à 3 achat de gros matériel. \\
Ceci résulte d'une politique d'extension et de renouvellement de matériels. Elle fixe les budgets annuels correspondant aux remplacements ou aux acquisitions de nouveaux matériels. Elle procède aux décisions d'investissement exceptionnel (remplacement urgent d'un matériel coûteux, extension du parc pour les besoins d'un nouveau chantier...)
La réalisation et les suivis des investissements concerne très peu les pièces de rechange



\subsection{Procédures d'achat:}

\begin{description}
      \item[Planification]~\\

Il y a deux types de planification : \\
la première étant celle qui programme l'utilisation des matériels par période et par type de matériel, cette planification s'appuie sur des prévisions d'utilisation et permet d'établir
une liste de besoins matériels é acquérir. Une demande d'acquisition est soumise au processus qui gère les investissements(cf plus bas)

Le deuxième type de planification est celui de l'affectation, è l'issue de laquelle, des demandes d'achat et des demandes de location sont générées:
\begin{itemize}
\item la procédure Acheter Matériel élabore des commandes d'achat de matériel
\item la procédure Louer Matériel élabore des commandes de location
\end{itemize}

Remarque : on voit que la planification ne considère pas les pièces de rechange mais uniquement les matériels. Les prévisions concernent en effet
les matériels. Néanmoins une planification d'achat des pièces de rechange pourrait ne pas être inintéressante.

    
    \item[Approvisionnement des pièces de rechange]~\\

L'approvisionnement des pièces de rechange se déroule suivant le cycle suivant :
\begin{itemize}
    \item abord on calcule le stock des pièces de rechange à chaque fois que l'on fait l'inventaire (ici annuel) ou lorsque l'on reçoit des pièces, ou lorsqu'une pièce sort du "magasin".
    \item On produit ainsi la variation de stock qui, associée aux prévisions de consommation nous conduit à calculer les besoins en pièces de rechange
    \item Enfin on transmet une demande de réapprovisionnement qui aura pour effet de produire une commande de pièces de rechange.
\end{itemize}

On peut également soumettre une demande de réapprovisionnement urgent sans passer par le calcul du stock, besoins ...

    \item[Renouvellement]~\\

Le département achat peut acquérir de nouveaux matériels sous certaines conditions (voir investissement)
ceci intervient après la planification de l'utilisation des matériels par type et concerne trés peu les pièces de rechanges.
\end{description}

\subsection{Etat du système informatique}
La gestion informatique au sein du département achat s'appuie sur deux types d'applications, la première étant des applications de gestion des fournisseurs (300 environ), la seconde des applications de gestion de bons de commande.
Au niveau de l'architecture, le département achat est équipé de deux ordinateur PC ainsi que de deux imprimantes.



\subsection{Dysfonctionnements}

\begin{itemize}
    \item Difficulté de trouver les critères d'investissement, achat vs location !
    \item Absence de politique d'investissement de pièces de rechange
    \item Absence de politique de stock de sécurité
    \item Achats groupés
    \item Absence d'étude prévisionnelle, simulation , statistique des matériels les plus utilisés, louer ce qu'on utilise le moins et acheter ce qu'on utilise le plus
    \item Absence de politique de choix de fournisseurs : actuellement elle s'appuie seulement sur la qualité et les prix, on pourrait envisager de rajouter le paramètre géographique, un fournisseur près d'un chantier ou des ateliers pourrait amortir les frais de transport (et même 'rapporter' beaucoup plus)


    \item Faire une étude des pièces qui se détériorent le plus et faire des commandes groupées a des fournisseurs intéressants


    \item Indépendance des applications et manque de communication entre elles.
\end{itemize}
