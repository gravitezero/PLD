\section{Gestion de la documentation du projet}

Pour accéder et gérer les documents, le responsable qualité à mis en place un dépôt Git hébergé sur le site GitHub.
Chacun des membres de l’équipe auras donc accès à tout moment aux documents pour les éditer en ligne, ou hors ligne.


Chaque document, sera rangé dans un dossier contenant ses sources au format texte, les documents joints, le livrable au format pdf, ainsi qu'un script permettant de créer le livrable facilement.

La production de document se fera à l’aide de LaTeX.


\section{Description des documents sur le plan de la forme}



\section{Le cycle de vie des documents}

Pour accélérer au maximum le projet, l’idéal est de minimiser le cycle de vie des documents.
Par exemple, ne pas rajouter d’étape supplémentaire de mise en page.

SCHEMA A INSERER %Je t'enverrai par mail le shcéma en PNG


\section{La procédure interne de validation}

La validation interne des livrables se fera de manière continue, tout au long de sa rédaction.
Lorsque l’auteur juge avoir fini tout ou partie du document, il contacte le responsable qualité pour une première relecture. Le chef de projet pourras effectuer une seconde relecture.

Si des corrections sont à faire, deux solutions sont possibles.
En cas de corrections mineurs (fautes d’orthographes, quelques problèmes de présentations ...) la correction est effectué directement par le relecteur.
En cas de correction majeur (problème de fond, irrespect du modèle de présentation ...) une demande de correction sera envoyé à l’auteur soulevant un problème ou en laissant des commentaires dans le document.

Dans tout les cas, les relectures se feront le plus vite possible.
L’auteur doit prévoir dans sa rédaction le temps nécessaire à la relecture et la correction afin d’éviter les retard dans la livraison du document.

Les documents qu'on va réaliser passent dans plusieurs états au fur et à mesure de leur élaboration. Chaque document peut passer par plusieurs états selon son contenu. Voici les états possibles pour un document réalisé:

TABLEAU A INSERER %Je te l'enverrai avec le png


\section{La procédure de recette client}

La procédure de validation du client pourras débuter à la date limite de rendu du livrable, sauf indication de retard. Le livrables sera déposé sur l'espace commun moodle dans la rubrique correspondante au document.

Si le client est dans l’incapacité de récupérer le document (document corrompu, introuvable ...) il peut prévenir par mail le responsable communication afin que l’équipe puisse redéposer le livrable.
Si moodle devenait inutilisable, les documents seraient livrés au client par mail temporairement.

Une fois le document en possession du client, la relecture pourra prendre jusqu’à une semaine.

En cas de non-validation, le client pourras contacter par mail ou pendant une réunion le responsable communication ou le chef d’équipe afin qu’il fasse remonter les remarques au reste de l’équipe.
Il sera en charge de formuler les réclamation de manière compréhensible.

En cas de correction à apporter au document, cette correction pourra être livré en 3 jours.
Nous serons chargé de prévenir le client de la livraison d’une correction.

\section{Les outils utilisés durant le projet}

La gestion de la documentation se fera à l’aide de l’outil Git, hébergé sur le site gIThUB.
GitHub pourras être utilisé pour remonter tout type de problème concernant la documentation.
La rédaction de document se fera à l’aide de n’importe quel éditeur de texte.
La production de la documentation se fera à l’aide de l'outil LaTeX.

Cette solution permettras à chacun  d’utiliser ces outils habituels d’éditions de textes, tout en assurant une cohérence parfaite entre les documents.

msproject
Pour prévoire les tâches à effectuer, et y affecter les ressources disponibles.

excel document de suivi



\section{ANNEXES}
Annexe 1 : Exemple de la page de garde
Modele
Annexe 2 : Fiche de suivi individuel par séance
Annexe 3 : Fiche global de suivi
Annexe 4 : Fiche de suivi d'avancement des livrables
Annexe 5 : Journal de réunions
Annexe 6 : Tableau de bord d'avancement
