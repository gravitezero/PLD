\section{Présentation de la solution spécifique}
La solution proposée s’appuie sur les services existants et vise à valoriser ces derniers. 
En effet, l’un des points fort de cette solution est de ne pas totalement bouleverser l’organisation actuelle afin de facilité sa mise en place.
Toutefois, afin de limiter les coûts logistiques quatre sites régionaux sont créés. 
Ils sont pilotés par un site central sur lequel le service achat et le nouveau service location travaillent en étroite collaboration. Chacun des sites (secondaires ou non) dispose de son propre parc matériel et magasin de pièces de rechange. Des ressources assurant la maintenance sont également affectées à chacune des antennes régionales. 
Il est important de préciser que la location du matériel reste une source de revenue secondaire, en effet, la priorité est donnée aux chantiers dont GSTP est propriétaire.
D’autre part des efforts conséquents sont réalisés en ce qui concerne l’informatisation du système complet :
    \begin{itemize}
        \item les sites et les chantiers sont équipés,
        \item un intranet est mis en place et une application spécifique est développée.
    \end{itemize}

En ce qui concerne l’application, l’un des atouts d’une solution spécifique est sa capacité d’adaptation aux besoins de GSTP. 
Ainsi, les modules planification/gestion du matériel, suivi des chantiers, achat/location, facturation et gestion des stocks optimisent les activités de la direction matérielle de GSTP et visent à répondre aux attentes des futurs utilisateurs.

En conclusion, la solution présentée ne remet pas totalement en cause l’organisation actuelle, optimise les coûts logistiques et facilite l’échange d’informations entre les différents services, les sites et les chantiers. 
De plus, le découpage de GSTP en sites régionaux permettra à la société de s’imposer comme un acteur local majeur.


\section{Plan de mise en oeuvre}
    \begin{itemize}
        \paragraph Phase 1
			\item Elaboration de la solution (9 mois)
				\subitem Etude Détaillée (1,5 mois)
				\subitem Développement de l’application spécifique (9 mois)
				\subitem Test (1,5 mois)
			\item Réorganisation de GSTP (9 mois)
				\subitem Achat des sous-sites 
				\subitem Affectation des nouvelles ressources
				\subitem Achat/Affectation du matériel et des pièces de rechange
				\subitem Informatisation progressive des sites et chantiers

		\paragraph Phase 2
			\item Déploiement architecture technique (2 mois)
				\subitem Installation réseau
				\subitem Déploiement de l'application (sites et chantiers)

		\paragraph Phase 3
			\item Accompagnement au changement (2 mois)
				\subitem Formation

		\paragraph Phase 4 
			\item Déploiement (1 mois)
				\subitem Migration des données existantes
				\subitem Qualification
				\subitem Mise en production

		\paragraph Phase 5
			\item Support et maintenance (contractuel  2 ans)
				\subitem Support téléphonique
				\subitem Suivi et Maintenance

            
        \item Total (parallélisme)
    \end{itemize}

\section{Evaluation des coûts}

    \subsection{Evaluation financière}
        \paragraph{Etienne}
        \begin{itemize}
            \item Informatique (Ordi, PC, PDA, ANtenne wifi, Serveur d'application / de données, imprimantes, routeurs, et abonnements)
        \end{itemize}
        
        \paragraph{Adrien}
        \begin{itemize}
            \item Immobilier
            \item Logisitique
            \item Sites secondaires
        \end{itemize}
        
        \paragraph{Thanh}
        \begin{itemize}
            \item Developpementspecifique SI
            \item Maintenance
            \item Formation
        \end{itemize}

\section{Evaluation des gains}

    \paragraph{Baptiste}
        \begin{itemize}
            \item Maintenance et immobilisation
            \item Location
            \item Efficacité Processus
            \item ...
            \item ...
            \item Total
        \end{itemize}
        
\section{Calcul du ROI}

\section{Evaluation risques solutions | delaid es satisfaction utilisateur | Maintenance Compétences externes}
