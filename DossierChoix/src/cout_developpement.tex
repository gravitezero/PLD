\section{Développement ERP}
\subsection{Objectif}
Dans le cadre du projet GSTP, afin de prévoir le coût, de mesurer la risque, ainsi que de faire des choix de solution, il faut on fasse une estimation sur le coût de développement logiciel.

La ``loi de Parkinson" indique bien que tout travail tend à
se dilater pour remplir tout le temps disponible, il est donc indispensable de l’estimer le plus exactement possible. 
 
\subsection{Méthode}

Etant donné que nous optons seulement pour le ``logiciels libres/open sources", le calcul ne sera basé que sur le charge de travail.\\

On décidé de le développement ne se sera effectué que par l’équipé interne sans rien externaliser, alors le coût se sera calculé basé sur l’évaluation durant cette phase, plus les frais engendrés par le consultant (si nécessaire dans le cas non-maitrisé). \\

L’unité de calcul sera ``jour*homme'' qui nous permettra également de rester cohérant dans le contexte du projet et de rendre plus flexible dans partage de travail.

\subsection{Estimation}
\subsubsection{Logiciels utilisés}
Pendant le développment, pour faciliter l'échange d'information, l'équipe développeurs doient utiliser les outils tels que: eGroupWare (groupe de travail), LAMP (pour le serveur), OpenOffice, Mozilla Thunderbird, Linux, Eclipse(poste de travail) qui sont touts gratuits.

\subsubsection{Développement}
La démarche utilisée est la classique cycle de V. 

\begin{enumerate}
\item \textbf{Elaboration:}\\ Il commencera par l'analyse de besoins, faisaibilité et la spécification ensemble. Cette tâche est confiée à une équipe de 4 personnes qui travail pendant 3 semaines soit 60 j*h. \\

En suite c'est la phase de spécification détaillée pour chaque module sur laquelle travaille 1 ou 2 personnes. Nous proposons une charge de Planification : 60+ Suivi de chantiers : 40+Achat/Location : 30+ Fracturation : 30+	Gestion de stocks : 30 soit 190 j*h.

\item \textbf{Construction}\\
Une fois la spécification est établie, il faut travailler sur la conception ensemble qui prendra 30j*h. 

\begin{center} 
    \begin{tabular}{ |c| c| c | c | c |c |}
    \hline
	 Module&Conception&Codage&Doc&Teste&Total\\ \hline
    Planification & 60 & 90 & 15 &35&210\\ \hline
    Suivi Chantier &45&90&10&25&170 \\ \hline
    Achat/Location &30&40&5&20&90  \\ \hline
    Fracturation &25&40&5&15&85\\ \hline
    Gestion de stock &30&45&5&20&80 \\
    \hline
    \end{tabular}
\end{center}

Suivant les tests unirataires c'est le test d'intégration pour valider que toutes les modules développées fonctionnent bien ensemble de façon cohérente, et éventuellement le test validation pour valider les exigences décrites par le clients. Les deux tests sont estimés à 30 j*h chacun, soit 60h.\\

Cette étape est envisage de se faire par les memebres de l'équipe de développement et de l'équipe de spécification. 

Il faut donc 700 j*h total pour cette phase de construction et validation.

\subsubsection{Déploiement}

Le déploiement se passera en général relativement vite avec la même équipe de développment. Il prendra environ 60 j*h.

\subsubsection{Maintenance}

L’entreprise GSDP pourrait devoir embaucher un informaticien expérimenté pour prendre en charge la maintenace le nouveau système (gérer les erreurs, assurer la performance, aider les utilisateurs). Son salaire est environ 50000 euro par an.\\

Le coût total de maintenance est très difficile à calcule à priori. 
Par contre, par expérience le coût annuel est approximent  10 \% de coût initial.

\subsection{Formation}
  La mise en place d'un système ERP nécessite une connaissance solide des futurs utilisateurs, la formation est donc une étape indispensable. Les utilisateurs ciblés sont les administratifs, les responsables et les techniciens. (il n'est pas nécessaire de former les ouvrieurs).\\

Les manuels d’utilisateur sont déjà créé dans la phase de développement.\\

En raison qu’il n'est impossible que tout le monde suive le cours en même temps, il faut faire un planning pour optimiser la perturbation de travail encadrée. \\
Il est proposé donc que soit pour chaque séance, il y a 3-5 administratifs avec  2-3 chefs de chantier et 6-8 techniciens supervisés par 3 formateurs expérimentés ou soit il sera se fait hors les horaires ouvriers (il faut alors les couts supplémentaires pour  les heurs hors contingent).\\
 
Toute fois, certaines parties ce la formation peuvent être effectuées par l'équipe informatique interne, qui réduit le coût.\\
 
L’estimation pour la formation est : 90 jour*homme

\end{enumerate}
