\section{Évaluation des livrables}
       \subsection{Dossier d'expression des besoins}
              \subsubsection{Étude de l'existant}
Nous avons mis en valeur l'existant organisationnel, informatique, les processus stratégiques (pour la facturation, l'approvisionnement, la demande matériel, la maintenance et la planification).Nous avons aussi mentionné les dysfonctionnements.

              \subsubsection{Benchmarking}
Nous avons procédé à un benchmarking non seulement au niveau des entreprises du même secteur d'activité (entreprises leaders du marché dans la construction), mais aussi au niveau des outils utilisés (ERP utilisés au sein des entreprises de BTP).

              \subsubsection{Thèmes de progrès}
Nous avons distingué les thèmes de progrès stratégique, organisationnel, fonctionnel, technologique.

       \subsection{Dossier de scenarii}
Nous avons proposé une architecture applicative avec un découpage en package, l'architecture logicielle et technique correspondaient également à nos prévisions. Le plus dans cette phase, est le choix de la mise en place d'un intranet et du déploiement de nos applications sur une plateforme web.

       \subsection{Dossier de choix}


       \subsection{Dossier de bilan}


\section{Bilan des charges}




\section{Synthèse des difficultés rencontrées}




\section{Bilans Personnels}

       \subsection{DIAKITE Naby Daouda (Chef de projet)}

       \subsection{BRODU Etienne (Responsable qualité)}

       \subsection{CHAZELLE JOHANN (Responsable communication)}

       \subsection{LECORNU Baptiste (Expert ERP/Modélisation)}

       \subsection{BACHATENE Chafik (Expert métier achat)}

       \subsection{PHAN DUC Thanh (Expert métier maintenance)}

       \subsection{BROCHOT Adrien (Expert métier gestion de matériels)}


\section{Annexes}

       \subsection{Diagramme de GANTT Final}


       \subsection{Tableau de bord avancement}





